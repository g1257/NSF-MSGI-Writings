\documentclass[10pt,reqno]{amsart}
\usepackage{amsfonts}
\usepackage{amscd}
\usepackage{amsmath}
\usepackage{amssymb}
\usepackage{amsthm}
\usepackage{enumitem}
\usepackage{pgf}
\usepackage{mathrsfs}
\usepackage[margin=1in]{geometry}
\usepackage{etoolbox}
\usepackage{hyperref}
\usepackage{soul}
\usepackage{todonotes}
\usepackage{braket}
\usepackage{dsfont}
\usepackage{tikz}
\usepackage{physics}
\usepackage[capitalise]{cleveref}
\patchcmd{\section}{\scshape}{\bfseries}{}{}
\makeatletter
\renewcommand{\@secnumfont}{\bfseries}
\makeatother

\newcommand{\Z}{\mathbb{Z}}
\newcommand{\N}{\mathbb{N}}
\newcommand{\ZZ}[1]{\mathbb{Z}/#1\mathbb{Z}}
\newcommand{\R}{\mathbb{R}}
\newcommand{\C}{\mathbb{C}}
\newcommand{\F}{\mathbb{F}}
\newcommand{\Q}{\mathbb{Q}}
\newcommand{\bP}{\mathbb{P}}
\newcommand{\RP}{\mathbb{RP}}
\newcommand{\D}{\mathcal{D}}
\newcommand{\cA}{\mathcal{A}}
\newcommand{\cB}{\mathcal{B}}
\newcommand{\cC}{\mathcal{C}}
\newcommand{\cF}{\mathcal{F}}
\newcommand{\cG}{\mathcal{G}}
\newcommand{\cO}{\mathcal{O}}
\newcommand{\cL}{\mathcal{L}}
\newcommand{\cM}{\mathcal{M}}
\newcommand{\cU}{\mathcal{U}}
\newcommand{\p}{\mathfrak{p}}
\newcommand{\q}{\mathfrak{q}}
\newcommand{\m}{\mathfrak{m}}
\newcommand{\ol}[1]{\overline{#1}}
\newcommand{\Mod}[1]{\ \left(\mathrm{mod}\ #1\right)}
\newcommand{\del}{\partial}
\newcommand{\sm}{\setminus}
\newcommand{\0}{\emptyset}
\newcommand{\cls}[1]{\overline{#1}}
\newcommand{\tsr}[1]{\otimes_{#1}}
\newcommand{\e}{\varepsilon}
\newcommand{\LNorm}[1]{\|#1\|_{L^2[-\pi,\pi]}}
\newcommand{\limup}[1]{\lim_{#1\rightarrow \infty}}
\newcommand{\M}[2]{\mathcal{M}_{#1\times #2}}
\newcommand{\IP}[2]{\langle#1,#2\rangle}
\newcommand*{\fixitem}[1]{\item[]
	\refstepcounter{enumi}\hskip-\labelwidth\hskip-\labelsep
	#1 \labelenumi}
\newcommand{\dbar}{{\mkern3mu\mathchar'26\mkern-12mu d}}

\newtheorem{theorem}{Theorem}
\newtheorem{proposition}[theorem]{Proposition}
\newtheorem{lemma}[theorem]{Lemma}
\newtheorem{conjecture}[theorem]{Conjecture}
\newtheorem{corollary}[theorem]{Corollary}

\theoremstyle{definition}
\newtheorem{example}[theorem]{Example}
\newtheorem{definition}[theorem]{Definition}

\theoremstyle{remark}
\newtheorem*{rem}{Remark}
\newtheorem*{note}{Note}
\newtheorem*{claim}{Claim}

\newcommand{\pder}[2]{\frac{\partial #1}{\partial #2}}
\newcommand{\secpder}[2]{\frac{\partial^2 #1}{\partial #2^2}}
\newsavebox{\smallblockbox}

\newenvironment{smallblockarray}
{\begin{lrbox}{\smallblockbox}
		\scriptsize$\begin{blockarray}}
	{\end{blockarray}$\end{lrbox}%
	\raisebox{-1ex}[\dimexpr\height-2ex][\dimexpr\depth-1ex]{\usebox{\smallblockbox}}}

\newcommand{\module}[2]{%
	\underset{\mathclap{\begin{smallmatrix}\\#2\end{smallmatrix}}}{#1}%
}

\DeclareMathOperator{\Aut}{Aut}
\DeclareMathOperator{\rad}{rad}
\DeclareMathOperator{\Inn}{Inn}
\DeclareMathOperator{\gap}{gap}
\DeclareMathOperator{\lgap}{lgap}
\DeclareMathOperator{\im}{im}
\DeclareMathOperator{\id}{id}
\DeclareMathOperator{\Jac}{Jac}
\DeclareMathOperator{\Int}{Int}
\DeclareMathOperator{\Ann}{Ann}
\DeclareMathOperator{\Ass}{Ass}
\DeclareMathOperator{\ch}{char}
\DeclareMathOperator{\Gal}{Gal}
\DeclareMathOperator{\Tor}{Tor}
\DeclareMathOperator{\Ext}{Ext}
\DeclareMathOperator{\Hom}{Hom}
\DeclareMathOperator{\End}{End}
\DeclareMathOperator{\Spec}{Spec}
\DeclareMathOperator{\ext}{ext}
\DeclareMathOperator{\supp}{supp}
\DeclareMathOperator{\coker}{coker}
\DeclareMathOperator*{\ES}{ES}
\DeclareMathOperator*{\esssup}{ess\,sup}
\DeclareMathOperator{\rk}{rk}
\DeclareMathOperator{\tr}{tr}
\DeclareMathOperator{\eig}{eig}
\DeclareMathOperator{\ev}{ev}
\DeclareMathOperator{\cl}{cl}
\DeclareMathOperator{\Ht}{ht}
\DeclareMathOperator{\inv}{inv}
\DeclareMathOperator{\adj}{adj}
\DeclareMathOperator{\des}{des}
\DeclareMathOperator{\maj}{maj}
\DeclareMathOperator{\Fr}{Fr}
\DeclareMathOperator{\Orb}{Orb}
\DeclareMathOperator{\Stab}{Stab}
\DeclareMathOperator{\Ofr}{OFr}
\DeclareMathOperator{\Map}{Map}
\DeclareMathOperator{\sign}{sign}
\DeclareMathOperator{\lcm}{lcm}
\DeclareMathOperator{\conv}{conv}
\DeclareMathOperator{\Res}{Res}
\DeclareMathOperator{\Log}{Log}
\DeclareMathOperator{\Pic}{Pic}
\DeclareMathOperator{\Div}{Div}
\DeclareMathOperator{\indeg}{indeg}
\DeclareMathOperator{\odeg}{outdeg}
\DeclareMathOperator{\Tr}{Tr}


%\newcommand{\norm}[1]{\left\lVert#1\right\rVert}
\makeatletter
\renewcommand*\env@matrix[1][*\c@MaxMatrixCols c]{%
	\hskip -\arraycolsep
	\let\@ifnextchar\new@ifnextchar
	\array{#1}}
\makeatother
\renewcommand{\Re}{\operatorname{Re}}
\renewcommand{\Im}{\operatorname{Im}}


\endinput


\begin{document}
	
	\title{Notes from NSF-MSGI Internship}
	\author{Hunter Lehmann}
	\maketitle
	
	%%%%%%%%%%%%%%%%%%%%%%%%%%%%%%%%%%%%%%%%%%%%%%%%%%%%%%%%%%%%%%%
	\section{Introduction}
	%%%%%%%%%%%%%%%%%%%%%%%%%%%%%%%%%%%%%%%%%%%%%%%%%%%%%%%%%%%%%%%
	
	We will explore the connection between statistical mechanics models and quantum Hamiltonians, primarily via examples. 
	
	More stuff here later. Basic objects/definitions.
	Mass gap/correlation length?
	
	%%%%%%%%%%%%%%%%%%%%%%%%%%%%%%%%%%%%%%%%%%%%%%%%%%%%%%%%%%%%%%%
	\section{Ising Models}
	%%%%%%%%%%%%%%%%%%%%%%%%%%%%%%%%%%%%%%%%%%%%%%%%%%%%%%%%%%%%%%%
	
	The Ising models are a family of statistical mechanics models with nearest-neighbor interaction. 
	Given any lattice $\cL \subset (a_i\Z)^d$ with $N$ total points for some scaling factors $a_i \in \R_{>0}$, define a \emph{spin} at each site of the lattice via a variable $s \in \Omega_0=\{\pm 1\}$. 
	A \emph{configuration} is a collection $\{s\}=\{s_\ell\}_{\ell\in\cL} \in \Omega_0^N$. 
	For convenience, we will usually work with cubic lattices of the form $\cL=(a_i\{-n,-n+1,\ldots, n-1,n\})^d$ or $\cL(=a_i\{1\ldots, n\})^d$ and impose periodic boundary conditions. 
	The \emph{action} associated to this model is the function 
	\[\mathcal{S}(\{s\},\tau, h)=-\beta_\tau(\tau)\sum_{i\sim j} s_is_j - \beta(\tau)h\sum_{i}s_i , \] 
	where $\tau$ is the lattice spacing in the ``time" direction, $h$ is the strength of an external magnetic field, and the first sum is over nearest neighbor sites of the lattice. 
	The coefficients $\beta_\tau$ and $\beta$ may depend on the time direction lattice spacing and are otherwise chosen to reflect some physical situation (e.g. they may be taken to be proportional to the ``inverse temperature" $\frac{1}{kT}$ where $k$ is Boltzmann's constant).
	
	We will be interested in the \emph{partition function} $Z$ defined as \[Z(\tau,h)=\sum_{\{s\}\in\Omega_0^N}e^{-\mathcal{S}(\{s\},\tau, h)}. \]
	
	We can use statistical mechanics to show that many physical quantities can be derived from the partition function \cite[Ch. 3]{friedli_velenik_2017}.
	The \emph{pressure in $\cL$} of the model is defined to be 
	\[\psi_{\cL}(\tau,h):=\frac{1}{N} \log Z(\tau,h).  \]
	The \emph{magnetization density in $\cL$} is, by definition,
	\[ m_\cL:=\frac{1}{N}\sum_{\ell \in \cL} s_\ell. \]
	The expected magnetization density, $\langle m_\cL \rangle$, is related to the pressure and partition function:
	\[\langle m_\cL \rangle (\tau,h) = \frac{1}{\beta}\frac{\partial \psi_{\cL}}{\partial h}(\tau, h)=\frac{\partial}{\partial h}\left( \frac{1}{\beta N} \log Z(\tau,h)\right) . \]
	
	
	%%%%%%%%%%%%%%%%%%%%%%%%%%%%%%%%%%%%%%%%%%%%%%%%%%%%%%%%%%%%%%%
	\subsection{Infinite Lattice Ising Models}
	
	The Ising model can be generalized to an \emph{infinite volume model} by letting the lattice $\cL$ grow to $\Z^d$ appropriately. 
	We'll briefly sketch the ideas here - a detailed exposition can be found in \cite{friedli_velenik_2017} in chapters 3 and 6.
	In general, we say a sequence of lattices $\{\cL_n \}$ converges to $\Z^d$, denoted $\cL_n \Uparrow \Z^d$, if 
	\begin{enumerate}
		\item $\cL_n$ is increasing, i.e. $\cL_n \subset \cL_{n+1}$ for all $n$,
		\item $\bigcup_{n} \cL_n = \Z^d$,
		\item $\lim\limits_{n\to \infty} \frac{|\partial \cL_n|}{|\cL_n|}=0$, where $\partial \cL_n$ is the boundary of the lattice, i.e. 
		\[\partial \cL_n := \{\ell \in \cL_n \mid \ell \sim m \, \text{for some $m$}\in \Z^d \sm \cL_n \}. \]
	\end{enumerate}
	It turns out that there is a well-defined limit 
		\[\psi(\tau,h):= \lim\limits_{\cL \Uparrow \Z^d} \psi_{\cL}(\tau,h).\]
	The resulting function $\psi$ is convex as a function from $\R \times \R_{>0}$ and even as a function of $h$.
	Because $\psi$ is a convex function of $h$, the \emph{average magnetization density} given by
		\[m(\tau,h)=\lim_{\cL \Uparrow \Z^d} \langle m_{\cL} \rangle (\tau, h), \]
	exists for all but a countable set of $h \in \R$.
	The points at which the average magnetization density fails to exist (always because the left and right derivatives do not equal each other) are called \emph{first-order phase transitions}.
		
	%%%%%%%%%%%%%%%%%%%%%%%%%%%%%%%%%%%%%%%%%%%%%%%%%%%%%%%%%%%%%%%
	\subsection{1+0 Ising}
	
	In the 1+0 Ising model, we take a 1-dimensional statistical mechanics system and relate it to a 0-dimensional (point) quantum Hamiltonian. See \cite{FradkinSusskind78,KogutGaugeSummary}.
	The definitions above simplify to the following. The lattice is now a set of $N$ points $x_0,\ldots,x_{N-1}$ on a circle so that $s_0=s_N$. The action is given by 
		\[\mathcal{S}(\{s\},\tau, h)=-\beta_\tau(\tau)\sum_{i=0}^{N-1} s_is_{i+1} - \beta(\tau)h\sum_{i=0}^{N-1}s_i. \] 
	It will be helpful to rewrite this expression to be symmetric and expressed in terms of sums and differences of the $s_i$. 
	Up to a constant factor of $-\beta_\tau N$, we find
		\[\mathcal{S}(\{s\},\tau)=\frac{1}{2}\beta_\tau(\tau)\sum_{i=0}^{N-1} (s_i-s_{i+1})^2 - \frac{1}{2}\beta(\tau)h\sum_{i=0}^{N-1}(s_i+s_{i+1}). \]
	Note that we needed periodicity in order to rewrite the second term in this way. Further, we now have $\mathcal{S}=0$ when all of the spins are identical and $h=0$.
	
	Now the partition function becomes 
	\begin{align*}
		Z&=\sum_{\{s\}\in\Omega_0^N} e^{-\frac{1}{2}\beta_\tau(\tau)\sum_{i=0}^{N-1} (s_i-s_{i+1})^2 + \frac{1}{2}\beta(\tau)h\sum_{i=0}^{N-1}(s_i+s_{i+1})} \\
		&=\sum_{s_0 \in \{\pm 1\}} \ldots \sum_{s_{N-1}\in \{\pm 1\}} e^{-\frac{1}{2}\beta_\tau (s_0-s_1)^2}e^{\frac{1}{2}\beta h(s_0+s_1)}\cdot\ldots\cdot e^{-\frac{1}{2}\beta_\tau (s_{N-1}-s_0)^2}e^{\frac{1}{2}\beta h(s_{N-1}+s_0)}.
	\end{align*}
	
	However, each factor in the product depends only on what the values of $s_i-s_{i+1}$ and $s_i+s_{i+1}$ are. 
	We can construct the \emph{transfer matrix}, $T$, to record this data, indexed by the possible spins at each site.
	So we have 
	\[T_{-1,-1}=e^{-\beta h}, \quad T_{-1,1}=T_{1,-1}=e^{-2\beta_\tau}, \quad \text{and}\ T_{1,1}=e^{\beta h}. \] 
	Thus \[T=\begin{bmatrix}
	e^{\beta h} & e^{-2\beta_\tau} \\
	e^{-2\beta_\tau} & e^{-\beta h}
	\end{bmatrix}. \]
	
	Now we can write $Z=\Tr T^N$.
	%TODO: expand on this claim
	Because $T$ is diagonalizable, we can write $T=SDS^{-1}$ and so 
		\[\Tr T^N=\Tr(SD^NS^{-1})=\Tr(D^N)=\lambda_0^N+\lambda_1^N,\]
	where $\lambda_0,\lambda_1$ are the eigenvalues of $T$.
	
	Via a bit of algebra, we see that 
	\begin{align*}
		\lambda_0 &=\cosh(\beta h)+ \sqrt{e^{-4\beta_\tau} +\sinh^2(\beta h) } \\
		\lambda_1 &=\cosh(\beta h)- \sqrt{e^{-4\beta_\tau} +\sinh^2(\beta h) }
	\end{align*} 
	Later, it will be clear why $\beta_\tau$ and $\beta$ should be taken as functions of $\tau$, and
	also why the dependence on $tau$ that we want is $\beta_\tau=-\frac{1}{2}\log \tau$ and $\beta = \tau$.
	In this case,
	\begin{align*}
		\lambda_0 &=\cosh(\tau h)+ \sqrt{\tau^2 +\sinh^2(\tau h)} \\
		\lambda_1 &=\cosh(\tau h)- \sqrt{\tau^2 +\sinh^2(\tau h)}
	\end{align*}
	
	Now the average magnetization is
	\begin{align*}
		m(\tau, h)&=\lim_{N\to \infty}\frac{\del}{\del h}\left( \frac{1}{\beta N}\log(\lambda_0^N+\lambda_1^N) \right) \\
			&=\frac{\del}{\del h}\lim_{N\to \infty}\left( \frac{1}{\beta N}\left(\log(\lambda_0^N)+\log(1+\left(\frac{\lambda_1}{\lambda_0}\right)^N)\right) \right) \\
			&= \frac{1}{\beta}\frac{\del}{\del h} \log(\lambda_0),
	\end{align*}
	since $|\lambda_1/\lambda_0| <1.$
	In our example case,
	\begin{align*}
		m(\tau,h)&= \frac{1/\tau}{\cosh(\tau h)+ \sqrt{\tau^2 +\sinh^2(\tau h)}}\left( \tau \sinh(\tau h)+\frac{\tau \sinh(\tau h)\cosh(\tau h)}{\sqrt{\tau^2 +\sinh^2(\tau h)}} \right)\\
		&=\frac{\sinh(\tau h)}{\sqrt{\tau^2+\sinh(\tau h)}}.
	\end{align*}

	We can now compute the limit of the expected average magnetization as we take the lattice distance $\tau$ to 0.
		\[ \lim_{\tau \to 0} m(\tau,h) = \lim_{\tau \to 0} \frac{\sinh(\tau h)}{\sqrt{\tau^2+\sinh(\tau h)}} = \frac{h}{\sqrt{1+h^2}}. \]
		
	From these computations we can see that $m(\tau,h)$ is a smooth function of $h$ and so the 1+0 Ising model does not have a (first-order) phase transition.
	
	%%%%%%%%%%%%%%%%%%%%%%%%%%%%%%%%%%%%%
	\subsubsection{Deriving the quantum Hamiltonian:}
	
	Our goal is to choose functions $\beta_\tau(\tau)$ and $\beta(\tau)$ so that the transfer matrix operator $T$ has the form $T=e^{-\tau H} \approx I-\tau H$ for some quantum Hamiltonian $H$ (independent of $\tau$) acting on a $2$-dimensional vector space. 
	This is the \emph{$\tau$-continuum Hamiltonian} for the model. 
	Here the idea is that statistical mechanics properties of the lattice action system (e.g. magnetization per site, average magnetization, two-point correlations, correlation length) will map to properties of the operator $H$ related to its eigenvalues and eigenvectors.
	
	For example, we could choose $\beta_\tau(\tau)=-\frac{1}{2}\log \tau$ and $\beta(\tau)=\tau$. Then we have \[T=\begin{bmatrix}
	e^{\tau h} & \tau \\
	\tau & e^{-\tau h}
	\end{bmatrix} \approx I_2-\tau\begin{bmatrix}
	-h & -1 \\
	-1 & h
	\end{bmatrix}.\] 
	From this it follows that \[H=\begin{bmatrix}
	-h & -1 \\
	-1 & h
	\end{bmatrix}=-\sigma_1-h\sigma_3, \] where $\sigma_1,\sigma_2$, and $\sigma_3$ are the Pauli matrices. 
	
	More generally, we can ask what constraints $\beta$ and $\beta_\tau$ must satisfy to be able to do this. 
	Analyzing each of the four matrix entries shows that for small $\tau$ we need 
	\[e^{\beta h} \approx 1-\tau H_{0,0}, \quad e^{-2\beta_\tau} \approx -\tau H_{1,0}=-\tau H_{0,1},\quad \text{and} \ e^{-\beta h} \approx 1- \tau H_{1,1}. \]
	
	Since $H_{1,0}$ and $H_{0,1}$ are constant with respect to $\tau$ we need $-2\beta_\tau \approx \log (\lambda\tau)$ for small $\tau$ and some $\lambda\in \R_{>0}$. 
	Similarly, if $\beta$ is small when $\tau$ is small we have $e^{\beta h} \approx 1-\beta h \approx 1-\tau H_{0,0}$ and $e^{-\beta h} \approx 1+\beta h \approx 1-\tau H_{1,1}$. 
	So to first order we have $\beta \approx \tau$ and we see $H_{0,0}=-H_{1,1}=-\mu h$ for some $\mu \in \R\sm\{0\}$. 
	
	Putting all of this together, we see that the general $\tau$-continuum Hamiltonian for this model will have the form 
	\[ H=\begin{bmatrix}
	-\mu h & -\lambda \\
	-\lambda & \mu h
	\end{bmatrix}=-\lambda\sigma_1-\mu h\sigma_3.\]
	
	Now we could also attempt to find a second order approximation of $T$ so that $T\approx I - \tau H + \frac{1}{2}\tau^2 H^2$ for an operator $H$ not depending on $\tau$. 
	However, following the approach above will show us that this is not possible in general.  
	Rather, to get a higher order approximation (or the actual matrix logarithm solving $T=e^{-\tau H}$ for $H$) we may need to allow $H$ to depend on $\tau$. 
	
	In the second order case, we need 
	\begin{align*}
		e^{\beta h} &\approx 1- \tau H_{0,0}+\frac{\tau^2}{2}(H_{0,0}^2+H_{0,1}^2),\\
		e^{-2\beta_\tau} &\approx -\tau H_{0,1}+\frac{\tau^2}{2}(H_{0,0}H_{0,1}+H_{0,1}H_{1,1}),\\
		e^{-\beta h} &\approx 1 - \tau H_{1,1}+\frac{\tau^2}{2}(H_{0,1}^2+H_{0,0}^2)
	\end{align*}
	for small $\tau$, where we have used the fact that $H$ should be real and symmetric and so $H_{0,1}=H_{1,0}$.
	
	From the first and third equations, we see that $H_{0,0}$ and $H_{1,1}$ must depend on $h$. 
	But the second equation has no dependence on $h$, so the dependence of $H_{0,1}$ on $h$ must be inversely related to the $H_{0,0}$ and $H_{1,1}$ dependence.
	On the other hand, adding the first and third equations yields
	\[\cosh(\beta h) \approx 1-\frac{\tau}{2}( H_{0,0}+ H_{1,1})+\frac{\tau^2}{4}(H_{0,0}^2+2H_{0,1}^2+H_{1,1}^2), \]
	which naturally suggests  taking $\beta=\mu\tau$, $H_{0,0}=-H_{1,1}=\mu h$, and $H_{0,1}=H{1,0}=0$. But this is a contradiction.
	
	Let's return to the running example where our Hamiltonian is now $H=-\sigma_1-h\sigma_3$.
	The eigenvalues of $H$ are $\lambda_0=-\sqrt{1+h^2}$ and $\lambda_1=\sqrt{1+h^2}$.
	The corresponding normalized eigenvectors are
		\[v_0= \frac{1}{\sqrt{2(1+h^2+h\sqrt{1+h^2})}}\begin{pmatrix}
		h+\sqrt{1+h^2} \\ 1
		\end{pmatrix} \, \text{and} \, v_1= \frac{1}{\sqrt{2(1+h^2-h\sqrt{1+h^2})}}\begin{pmatrix}
		h-\sqrt{1+h^2} \\ 1
		\end{pmatrix}. \]
	Now the ground state is $\ket{gs}=v_0$ and if we compute $\bra{gs} \sigma_3 \ket{gs}$, we get
		\[\bra{gs} \sigma_3 \ket{gs} \rangle=\frac{1}{2(1+h^2+h\sqrt{1+h^2})}\begin{pmatrix}
		h+\sqrt{1+h^2} & 1
		\end{pmatrix}
		\begin{pmatrix}
		1 & 0 \\
		0 & -1
		\end{pmatrix}
		\begin{pmatrix}
		h+\sqrt{1+h^2} \\ 1
		\end{pmatrix}=\frac{h}{\sqrt{1+h^2}} .\]
	
	This demonstrates the correspondence between properties of the original statistical mechanics model and observables derived from operators applied to the ground state of the Hamiltonian.
	
	%%%%%%%%%%%%%%%%%%%%%%%%%%%%%%%%%%%%%%%%%%%%%%%%%%%%%%%%%%%%%%%
	\subsection{1+1 Ising Model}
	
	We may analyze the $1+1$ Ising model similarly to the $1+0$ model. 
	We are going to relate a statistical mechanics system on a 2-dimensional lattice (with one temporal and one spatial dimension) to a quantum mechanical Hamiltonian on a 1-dimensional system of interacting spins.  
	See \cite{FradkinSusskind78,KogutGaugeSummary}.
	
	As in the 1+0 Ising model, we work with a lattice with periodic boundary conditions. 
	Suppose $\cL$ is a $N_x \times N_\tau$ lattice of points in $\Z^2$ with $\vec{\tau}$ a unit vector in the time direction and $\vec{x}$ a unit vector in the spatial direction. 
	Then the action for the model is 
	\[ \mathcal{S}= -\sum_{\ell} \beta_\tau s_{\ell}s_{\ell+\vec{\tau}}+\beta s_{\ell}s_{\ell+\vec{x}}, \]
	where the sum is over all lattice points $\ell \in \cL$.
	
	We want to write $\mathcal{S}=\sum_{j=1}^{N_\tau} L(j,j+1)$ for some $L(j,j+1)$ that describes the interaction between the spatial rows $j$ and $j+1$.
	To do this, we first rewrite 
	\[\mathcal{S}=\sum_{\ell}\frac{\beta_\tau}{2}(s_{\ell}-s_{\ell+\vec{\tau}})^2 -\frac{1}{2}\beta (s_\ell s_{\ell+\vec{x}}+s_{\ell+\vec{\tau}}s_{\ell+\vec{\tau}+\vec{x}}), \]
	
	using the periodic boundary conditions. 
	Note that the new $\mathcal{S}$ differs from the previous one by a normalization constant so that the first term is 0 when all of the spins are aligned, rather than being $-N\beta_\tau$.
	We can now define 
	\[L(j,j+1):= \sum_{\ell=1}^{N_x} \frac{\beta_\tau}{2} (s_\ell-\tilde{s}_\ell)^2-\frac{\beta}{2}(s_\ell s_{\ell+1}+\tilde{s}_\ell \tilde{s}_{\ell+1}), \]
	where the sum is over the $N_x$ indices in the spatial row, $\{s\}$ is the configuration of row $j$ and $\{\tilde{s}\}$ is the configuration of row $j+1$.
	
	Then the partition function is 
	\[ Z=\sum_{\{s\}} e^{-L(1,2)}e^{-L(2,3)}\ldots e^{-L(N_\tau,1)}. \]
	As in the previous section, we express $Z$ as the trace of the $N_\tau^\text{th}$ power of a transfer matrix $\hat{T}$ describing the transition between rows. 
	Since there are $2^{N_x}$ configurations for each row, $\hat{T}$ will be a $2^{N_x}\times 2^{N_x}$ matrix.
	The elements of $\hat{T}$ can be organized by the number of spin flips between configurations, since these determine the value of the first term of $L(j,j+1)$. We want to find $\beta,\beta_\tau$ so that for $\tau$ near 0, $\hat{T} \approx 1 - \tau \hat{H}$.
	\begin{align*}
		\hat{T}|_{0\,\text{flips}} &= e^{\beta \sum_{\ell=1}^{N_x} s_{\ell} s_{\ell+1}},  \\		
		\hat{T}|_{1\,\text{flip}} &= e^{-2\beta_\tau} e^{\frac{\beta}{2}\sum_{\ell=1}^{N_x}(s_\ell s_{\ell+1}+\tilde{s}_\ell \tilde{s}_{\ell+1})},\\
		\vdots \quad & \qquad\qquad\qquad \vdots \\		
		\hat{T}|_{k\,\text{flips}} &= e^{-2k\beta_\tau} e^{\frac{\beta}{2}\sum_{\ell=1}^{N_x}(s_\ell s_{\ell+1}+\tilde{s}_\ell \tilde{s}_{\ell+1})}		
	\end{align*}
	
	A simple solution is to choose $\beta =\lambda \tau$ and $\tau = e^{-2\beta_\tau}$ for some $\lambda \in \R_{>0}$.
	Then to first order approximation in $\tau$,
	\begin{align*}
		\hat{T}|_{0\,\text{flips}} &\approx 1-\mu\lambda\tau, \\
		\hat{T}|_{1\,\text{flip}} &\approx \tau (1+\kappa \lambda\tau) \approx \tau, \\
		\vdots \quad & \qquad\qquad \vdots \\		
		\hat{T}|_{k\,\text{flips}} &\approx \tau^k (1+\kappa \lambda\tau) \approx 0, 
	\end{align*}
    where $\mu=\sum_{\ell=1}^{N_x}s_{\ell}s_{\ell+1}$ and $\kappa=\frac{1}{2}\sum_{\ell=1}^{N_x} (s_{\ell}s_{\ell+1}+\tilde{s}_{\ell}\tilde{s}_{\ell+1})$ are independent of $\tau$.
    
    We can now express $\hat{T}$ as $I- \tau \hat{H}$ where the Hamiltonian $\hat{H}$ is given in terms of the Pauli operators at each site $\ell$, $\hat{\sigma}_1(\ell)$ and $\hat{\sigma}_3(\ell)$:
    \begin{equation}\label{eq:1+1IsingHamiltonian}
    \hat{H}= -\lambda \sum_{\ell=1}^{N_x} \hat{\sigma}_3(\ell)\hat{\sigma}_3(\ell+1) - \sum_{\ell=1}^{N_x} \hat{\sigma}_1(\ell). 
    \end{equation}
    
    %%%%%%%%%%%%%%%%%%%%%%%%%%%%%%%%%%%%%
    \subsubsection{Phase transition of 1+1 Ising model}
    
    There are a variety of ways to see that the 1+1 Ising model has a phase transition.
    (e.g. duality) \cite{KogutGaugeSummary}
    %TODO Fill in details here
    
    %%%%%%%%%%%%%%%%%%%%%%%%%%%%%%%%%%%%%
    \subsubsection{Jordan-Wigner transform and solution in terms of Fermionic operators}
    
    A Hamiltonian of the form of \cref{eq:1+1IsingHamiltonian} can be rewritten in terms of Fermion operators $\{a_j,a^\dagger_j\}_{j=1}^n$ that satisfy the \emph{canonical commutation relations} (CCRs) 
    \[ \{a_j,a_k^\dagger \} = \delta_{k,j}I; \qquad \{a_j,a_k \} =0, \]
    where $\{A,B\} = AB + BA$ is the anticommutator of two operators \cite{KogutGaugeSummary,nielsen_fermions,SchultzMattisLieb64}.
    
    We can summarize this transformation as follows:
    \begin{enumerate}
    	\item Use duality to swap the roles of $\hat{\sigma}_1$ and $\hat{\sigma}_3$ in \cref{eq:1+1IsingHamiltonian}.
    	\item Use raising and lowering operators to rewrite $H$ in terms of fermion operators. (Jordan-Wigner transform)
    	\item Convert the resulting operators and quadratic Hamiltonian to momentum space.
    	\item Diagonalize the momentum Fermionic Hamiltonian.
    	\item Determine the eigenvalues of the resulting Hamiltonian.
    \end{enumerate}
	
	%%%%%%%%%%%%%%%%%%%%%%%%%%%%%%%%%%%%%%%%%%%%%%%%%%%%%%%%%%%%%%%
	\section{O(N) Model}
	%%%%%%%%%%%%%%%%%%%%%%%%%%%%%%%%%%%%%%%%%%%%%%%%%%%%%%%%%%%%%%%
	
	These models are extensively treated in \cite{FradkinSusskind78,HamerKogutSusskind79,KogutGaugeSummary}. 
	As in the Ising model, we have a lattice $\cL \subset \Z^d$ of $M$ points, usually cubic. 
	The spins at each site $i$ of the lattice are now $\vec{x_i} \in \Omega_0=S^{N-1} = \{\vec{x} \in \R^{N} \mid \|\vec{x}\|=1 \}$. 
	The action is
	\[\mathcal{S}=\sum_{i,j} -J_{i,j}\vec{x_i}\cdot\vec{x_j}, \] where the sum is over nearest-neighbor pairs in $\cL$.
	Frequently we will take the interaction constants $J_{i,j}$ to depend only on which coordinate of the lattice the points differ in.
	As in the earlier examples, it may be helpful to renormalize the action so that when all the $\vec{x_i}$ are equal we get $\mathcal{S}=0$. 
	This results in a renormalized action
	\[\mathcal{S}=\sum_{i,j}\frac{1}{2} J_{i,j}(\vec{x_i}-\vec{x_j})^2. \]
	
	As before, we are interested in thinking of our $d$-dimensional lattice as having $1$ time dimension and $d-1$ spatial dimensions and finding a $\tau$-continuum quantum mechanical Hamiltonian $H$ that corresponds to the action as the lattice spacing $\tau$ goes to $0$.
	Since the spin configuration space is now continuous, the partition function will now be an integral rather than a summation:
	\[ Z=\int_{(S^N)^M} d\vec{x}_1\ldots d\vec{x_M} e^{-\beta\sum_{i,j}\frac{1}{2} J_{i,j}(\vec{x_i}-\vec{x_j})^2}. \]
	
	%%%%%%%%%%%%%%%%%%%%%%%%%%%%%%%%%%%%%%%%%%%%%%%%%%%%%%%%%%%%%%%
	\subsection{O(2) model on a 1-dimensional lattice}
	
	First we consider the $O(2)$ model on a 1-dimensional lattice of $N$ points with periodic boundary conditions and a magnetic field of strength $h$.
	Since our configuration space is $S^1$, we can parameterize the spins $\vec{x_i}=(\cos(\theta_i),\sin(\theta_i))$ for $\theta \in [0,2\pi)$. 
	Under this parameterization, the action becomes
	\begin{align*}
		 \mathcal{S} &= -\beta_\tau \sum_{j=1}^N \cos(\theta_j)\cos(\theta_{j+1})+\sin(\theta_j)\sin(\theta_{j+1}) - \beta h  \sum_{j=1}^N \cos(\theta_j)\\
			&=  - \beta_\tau \sum_{j=1}^N \cos(\theta_j-\theta_{j+1}) - \frac{\beta h}{2}  \sum_{j=1}^N \left(\cos(\theta_j)+\cos(\theta_{j+1})\right). 
	\end{align*}
	
	The corresponding partition function is then
	\[ Z = \int_{[0,2\pi]^N} \left(\prod_{j=1}^N d\theta_j \right) e^{\beta_\tau \sum_{j=1}^N \cos(\theta_j-\theta_{j+1})} e^{\frac{\beta h}{2}  \sum_{j=1}^N \cos(\theta_j)+\cos(\theta_{j+1})}. \]
	We now wish find an analogue of the transfer matrix from the analysis of the Ising model.
	To do this, we need the concept of integral operators.
	Given any $L^2$ function $f(x,y)$ on a domain $E\times E$, we can define the operator $L_f : L^2(E) \to L^2(E)$ by 
	\[(L_f g)(x)=\int_E f(x,y)g(y)\, dy \]
	for any $g \in L^2(E)$.
	
	Define $f(\theta,\phi): [0,2\pi]\times [0,2\pi]$ by 
	\[f(\theta,\phi)=e^{\beta_\tau \cos(\theta -\phi)}e^{\frac{\beta h}{2}(\cos(\theta)+\cos(\phi))}. \]
	Then the integral operator $\hat{T}=L_f$ plays the same role as the transfer matrix above.
	Since $f$ is in $L^2([0,2\pi]^2)$, $\hat{T}$ is trace-class and we have
	\[Z= \int_{[0,2\pi]^N} \left(\prod_{j=1}^N d\theta_j \right) f(\theta_1,\theta_2)\ldots f(\theta_N,\theta_1) = \Tr \hat{T}^N. \]
	
	Again, our new goal is to write $\hat{T}=I-\tau\hat{H}$ when $\tau$ is small for some Hamiltonian on the one site Hilbert space $L^2([0,2\pi])$ without needing to solve the problem exactly. 
	Recall that the set $B=\{\psi_m:=\frac{1}{\sqrt{2\pi}} e^{im\theta} \mid m \in \Z \}$ forms an orthonormal Hilbert basis for $L^2([0,2\pi])$.
	We will approximate the action of $\hat{T}$ on elements $\psi_m$ of this basis in order to find our approximate Hamiltonian $\hat{H}$.
	\begin{align}
		(\hat{T}\psi_m)(\theta) &= \frac{1}{\sqrt{2\pi}}\int_{0}^{2\pi} e^{\beta_\tau \cos(\theta -\phi)}e^{\frac{\beta h}{2}(\cos(\theta)+\cos(\phi))} e^{im\phi}\, d\phi \nonumber \\
		&=\frac{1}{\sqrt{2\pi}}e^{\frac{1}{2}\beta h \cos(\theta)} \int_{0}^{2\pi} e^{\beta_\tau \cos(\theta -\phi)}e^{\frac{1}{2}\beta h \cos(\phi)}e^{im\phi}\, d\phi. \label{eq:1+0_O(2)_IntOperator}
	\end{align}
	
	We can use a Fourier transform to rewrite the $\cos(\theta-\phi)$ portion of the exponential in \cref{eq:1+0_O(2)_IntOperator}.
	\begin{equation}\label{eq:FourierTransformExpCos}
		e^{-\beta_\tau+\beta_\tau\cos(\theta-\phi)}=\sum_{\ell\in \Z}e^{i\ell(\theta-\phi)}I_\ell(\beta_\tau),
	\end{equation}
	where $I_\ell(\beta)$ is the Bessel function of imaginary argument.
	In particular, we will let $\beta_\tau=\tau^{-1}$ so that $\beta_\tau$ is large when $\tau$ goes to 0. 
	Then we can approximate $I_\ell(\beta_\tau)$ by the Gaussian $e^{-\ell^2/2\beta_\tau}$ and \cref{eq:FourierTransformExpCos} becomes
	\begin{align}
		e^{\beta_\tau\cos(\theta-\phi)}&\approx e^{\beta_\tau}\sum_{\ell\in \Z}e^{i\ell(\theta-\phi)}e^{-\ell^2/2\beta_\tau}\\
			&=e^{1/\tau}\sum_{\ell\in \Z}e^{i\ell(\theta-\phi)}e^{-\tau\ell^2/2}. \label{eq:FourierTransformExpCosApprox}
	\end{align}
	
	After substituting \cref{eq:FourierTransformExpCosApprox} into \cref{eq:1+0_O(2)_IntOperator}, we can use absolute convergence of the sum to rewrite the expression further.
	\begin{align}
		(\hat{T}\psi_m)(\theta) &\approx \frac{1}{\sqrt{2\pi}}e^{\frac{1}{2}\beta h \cos(\theta)} \int_{0}^{2\pi} e^{1/\tau} \left( \sum_{\ell\in \Z}e^{i\ell(\theta-\phi)}e^{-\tau\ell^2/2} \right) e^{\frac{1}{2}\beta h \cos(\phi)}e^{im\phi}\, d\phi \nonumber \\
			&= \frac{1}{\sqrt{2\pi}}e^{\frac{1}{2}\beta h \cos(\theta)+1/\tau} \sum_{\ell\in \Z} e^{i\ell\theta-\tau\ell^2/2}\left(\int_{0}^{2\pi} e^{\frac{1}{2}\beta h \cos(\phi)}e^{i(m-\ell)\phi}\right) \, d\phi. \label{eq:1+0_O(2)_ApproxIntOperator}
	\end{align}
	
	At this point, we want $\beta\beta_\tau$ to remain finite as $\tau$ goes to 0, so $\beta=\lambda \tau$ for some constant $\lambda$. 
	Thus $\frac{1}{2}\beta h \cos(\phi)$ goes to 0 as $\tau$ goes to 0 and we can expand the corresponding exponential in \cref{eq:1+0_O(2)_ApproxIntOperator}.
	We then use orthogonality to evaluate the resulting integral.
	\begin{align*}
		(\hat{T}\psi_m)(\theta) &\approx \frac{1}{\sqrt{2\pi}}e^{\frac{\lambda\tau h}{2} \cos(\theta)+1/\tau} \sum_{\ell\in \Z} e^{i\ell\theta-\tau\ell^2/2}\left(\int_{0}^{2\pi} (1+\left(\frac{\lambda \tau h}{2}\cos(\phi) \right) e^{i(m-\ell)\phi}\right) \, d\phi,  \\
		&= \frac{1}{\sqrt{2\pi}}e^{\frac{\lambda\tau h}{2} \cos(\theta)+1/\tau} \sum_{\ell\in \Z} e^{i\ell\theta-\tau\ell^2/2}\left( 2\pi \delta_{\ell,m}+ \frac{\lambda \tau h}{2}\int_{0}^{2\pi} \left( \frac{e^{i\theta}+e^{-i\theta}}{2} \right)  e^{i(m-\ell)\phi}\right) \, d\phi,  \\
		&= \frac{1}{\sqrt{2\pi}} e^{\frac{\lambda\tau h}{2} \cos(\theta)+1/\tau} \sum_{\ell\in \Z} 2\pi e^{i\ell\theta-\tau\ell^2/2}\left(\delta_{\ell,m}+ \frac{\lambda \tau h}{4}(\delta_{m+1,\ell}+\delta_{m-1,\ell}) \right),  \\
		&= \sqrt{2\pi} e^{\frac{\lambda\tau h}{2} \cos(\theta)+1/\tau} ( e^{im\theta - \tau m^2/2}+ \frac{\lambda \tau h}{4} (e^{i(m+1)\theta - \tau (m+1)^2/2}+ e^{i(m-1)\theta - \tau (m-1)^2/2} ) ). 
	\end{align*}
	
	At this point, we need to rescale the problem to drop the factor of $e^\frac{1}{\tau}$, as this term goes to infinity as $\tau$ goes to 0.
	After we do this, we will write the action of $\hat{T}$ on $\psi_m$ in terms of the operators $J_z$ and $J_{\pm}$ defined on our basis and extended linearly by 
		\[J_z \psi_m = m \cdot \psi_m, \quad J_{\pm} \psi_m = \psi_{m\pm1}. \]
	To do this we use the fact that $\tau$ is small to expand the remaining exponentials that do not depend on $\theta$ and drop all terms involving a power of $\tau$ greater than one.
	\begin{align}
		(\hat{T}\psi_m)(\theta) &\approx \sqrt{2\pi} \left( 1+\frac{1}{2}\lambda\tau h \cos(\theta)\right) \left( \left( 1- \frac{1}{2}\tau m^2\right) e^{im\theta}+ \frac{\lambda \tau h}{4} \left( e^{i(m+1)\theta}+ e^{i(m-1)\theta} \right) \right) ,\nonumber\\
		&\approx\sqrt{2\pi} \left( \left( 1- \frac{1}{2}\tau m^2+\frac{1}{2}\lambda\tau h \cos(\theta)\right)  e^{im\theta}+ \frac{\lambda \tau h}{4} \left( e^{i(m+1)\theta}+ e^{i(m-1)\theta} \right) \right),\nonumber\\
		&=\sqrt{2\pi} \left( \left( 1- \frac{1}{2}\tau m^2\right)  e^{im\theta}+ \frac{\lambda \tau h}{2} \left( e^{i(m+1)\theta}+ e^{i(m-1)\theta} \right) \right),\nonumber\\
		&=\left( 2\pi \left( I-\tau \left( \frac{J_z^2}{2} - \frac{\lambda h}{2}(J_+ + J_-)\right) \right) \psi_m \right)(\theta).\label{eq:1+0_O(2)_Op_1-tH_Form}
	\end{align}
	
	From \cref{eq:1+0_O(2)_Op_1-tH_Form} we see that the desired Hamiltonian is 
	\begin{equation}\label{eq:1+0_O(2)_Hamiltonian}
		\hat{H}=2\pi \left( \frac{1}{2}J_z^2- \frac{\lambda h}{2} (J_+ + J_-) \right).
	\end{equation}
	
	%%%%%%%%%%%%%%%%%%%%%%%%%%%%%%%%%%%%%%%%%%%%%%%%%%%%%%%%%%%%%%%
	\subsection{O(2) model on a 2-dimensional lattice}
	
	We can perform a similar analysis on a 2-dimensional $M\times N$ lattice where there are $M$ spatial rows and $N$ temporal columns. 
	For simplicity, we will drop the magnetic field term and consider only the interactions between neighboring spins. 
	We will fix all interaction constants in the temporal direction (with unit vector $\vec{\tau}$ and temporal lattice spacing $\tau$) to be some $\beta_\tau(\tau)$.
	The interaction constants in the spatial direction (with unit vector $\vec{x}$) will be $\beta(\tau)$.
	For brevity we will omit the explicit dependence on $\tau$.
	
	The action then becomes
		\[S=-\beta_\tau\sum_{\ell}\cos(\theta_\ell-\theta_{\ell+\vec{\tau}})-\beta\sum_{\ell}\cos(\theta_\ell-\theta_{\ell+\vec{x}}). \]
	
	It will be helpful later to adjust the action by a constant factor of $MN\beta_\tau-\frac{MN}{2}\log(2\pi\beta_\tau)$.
	We also write $\theta_j^k$ for the angle in row $k$ and column $j$ of the lattice and rewrite the second term to be symmetric between adjacent rows.
	The resulting action is
		\[ S=\sum_{k=1}^M L(\vec{\theta}^k,\vec{\theta}^{k+1}) ,\]
	where
		\[ L(\vec{\theta},\vec{\phi})=-\frac{N}{2}\log(2\pi\beta_\tau) -\beta_\tau\sum_{j=1}^N (-1+\cos(\theta_j-\phi_j)) -\frac{\beta}{2} \sum_{j=1}^N \left(  \cos(\theta_j-\theta_{j+1}) + \cos(\phi_j-\phi_{j+1}) \right)\]
	
	Then 
		\[ Z=\int d\vec{\theta}^1 \ldots d\vec{\theta}^M e^{-L(\vec{\theta}^1,\vec{\theta}^2)}\ldots e^{-L(\vec{\theta}^N,\vec{\theta}^1)}=\Tr(\hat{T}^M), \]
	where $\hat{T}$ is the integral operator on the Hilbert space $L^2([0,2\pi]^N)\cong L^2([0,2\pi])^{\otimes N}$ defined by 
		\[(\hat{T}f)(\vec{\theta})=\int_{[0,2\pi]^N} e^{-L(\vec{\theta},\vec{\phi})}f(\vec{\phi}) \, d\vec{\phi}.\]
	
	As in the previous section, we will analyze the action of $\hat{T}$ on elements of the orthornomal Hilbert basis 
		\[ B=\left\{\psi_{\vec{m}}(\vec{\theta}) := \frac{1}{(2\pi)^{N/2}}\prod_{j=1}^N e^{im_j\theta_j} \mid \vec{m} \in \Z^N \right\} \]
	of $L^2([0,2\pi]^N)$.
	This will go very similarly and we will need the following approximations for small $\tau$ and $m\in \Z$.
	We will also take $\beta_\tau=1/\tau$ and $\beta=\lambda\tau$ for some constant $\lambda$.
	\begin{align}
		\int_0^{2\pi} e^{\frac{1}{\tau}(-1+\cos(\theta-\phi))}e^{im\phi} \, d\phi &\approx \frac{\sqrt{\tau}}{\sqrt{2\pi}} \int_0^{2\pi}\sum_{\ell \in \Z} e^{i\ell (\theta -\phi)-\ell^2\tau/2+im\phi}\, d\phi = \sqrt{2\pi \tau}e^{-m^2\tau/2} e^{im\theta}.\label{eq:O2_bessel_approx_int} \\
		e^{\frac{1}{2}\lambda\tau \sum_{j=1}^N \cos(\phi_j-\phi_{j+1})}&\approx 1+\frac{\lambda\tau}{4}\sum_{j=1}^N (e^{i\phi_j}e^{-i\phi_{j+1}}+e^{-i\phi_j}e^{i\phi_{j+1}}) \label{eq:O2_exp_cos_expansion}
	\end{align} 
	
	We will write our Hamiltonian in terms of the operators $J_z(j), J_+(j)$, and $J_-(j)$ defined on the basis $\{\psi_{\vec{m}} \}$ by
	\begin{align*}
		(J_z(j)\psi_{\vec{m}})(\vec{\theta}) &:= m_j \psi_{\vec{m}}, \\
		(J_{\pm}(j)\psi_{\vec{m}})(\vec{\theta}) &:= e^{\pm i \theta_j}\psi_{\vec{m}}.
	\end{align*} 
	
	Now 
	\begin{align*}
		(\hat{T}\psi_{\vec{m}})(\vec{\theta})&=\frac{1}{(2\pi)^{N/2}}\int_{[0,2\pi]^N} e^{-L(\vec{\theta},\vec{\phi})}e^{\sum_{j=1}^N im_j\phi_j} \, d\vec{\phi}. \\
		&= F(\vec{\theta}) \int e^{\frac{1}{\tau}\sum_{j=1}^N (-1+\cos(\theta_j-\phi_j)+im_j\phi_j) +\frac{\lambda\tau}{2} \sum_{j=1}^N \cos(\phi_j-\phi_{j+1}) }\, d\vec{\phi},
	\end{align*}
	where $F(\vec{\theta})=\left(\frac{1}{\tau}\right)^{N/2}e^{\frac{\lambda\tau}{2}\sum_{j=1}^N\cos(\theta_j-\theta_{j+1})}$.
	We can now use \cref{eq:O2_bessel_approx_int} and \cref{eq:O2_exp_cos_expansion}.
	Whenever we take an expansion in powers of $\tau$ we will neglect any terms with order at least 2.
	\begin{align*}
		(\hat{T}\psi_{\vec{m}})(\vec{\theta})& \approx F(\vec{\theta}) \int e^{\frac{1}{\tau}\sum_{j=1}^N (-1+\cos(\theta_j-\phi_j)+im_j\phi_j)}\left( 1+\frac{\lambda\tau}{4}\sum_{j=1}^N (e^{i\phi_j}e^{-i\phi_{j+1}}+e^{-i\phi_j}e^{i\phi_{j+1}}) \right)\, d\vec{\phi},\\
		&\approx F(\vec{\theta})\left( \prod_{j=1}^N \sqrt{2\pi \tau}e^{-m_j^2\tau/2} e^{im_j\theta_j} +\sum_{\substack{1\leq k,\ell \leq N\\k\neq \ell}}\prod_{j=1}^N \sqrt{2\pi \tau} e^{i(m_j+\delta_{k,j}-\delta_{\ell,j})\theta_j}e^{-\tau(m_j+\delta_{k,j}-\delta_{\ell,j})^2/2} \right),\\
		&\approx F(\vec{\theta})\left( \prod_{j=1}^N \sqrt{2\pi \tau}(1-m_j^2\tau/2) e^{im_j\theta_j} +\sum_{k\neq \ell}\prod_{j=1}^N \sqrt{2\pi \tau}(1-\tau(m_j+\delta_{k,j}-\delta_{\ell,j})^2/2) e^{i(m_j+\delta_{k,j}-\delta_{\ell,j})\theta_j} \right),\\
		&\approx F(\vec{\theta})(2\pi\sqrt{\tau})^N \left( \left(I-\frac{\tau}{2}\sum_{j=1}^N J_z^2(j) +\frac{\lambda\tau}{4} \sum_{j=1}^N (J_+(j)J_-(j+1)+J_-(j)J_+(j+1))\right) \right) \psi_{\vec{m}}(\vec{\theta}),\\
		&\approx (2\pi)^N  \left(I-\frac{\tau}{2}\sum_{j=1}^N J_z^2(j) -\lambda (J_+(j)J_-(j+1)+J_-(j)J_+(j+1))\right) \psi_{\vec{m}}(\vec{\theta}).
	\end{align*}
	
	It now follows that up to a multiplicative constant the desired Hamiltonian is 
		\[ \hat{H}:=\frac{1}{2} \sum_{j=1}^N J_z^2(j) -\lambda (J_+(j)J_-(j+1)+J_-(j)J_+(j+1)). \]
	
	
	%%%%%%%%%%%%%%%%%%%%%%%%%%%%%%%%%%%%%%%%%%%%%%%%%%%%%%%%%%%%%%%
	\section{(Mean) Spherical Model}
	%%%%%%%%%%%%%%%%%%%%%%%%%%%%%%%%%%%%%%%%%%%%%%%%%%%%%%%%%%%%%%%
	
	Detailed analysis of the spherical model in various dimensions can be found in \cite{HenkelHoeger84,Thompson68}. 
	We will sketch the approach of \cite{HenkelHoeger84} to determining a quantum Hamiltonian for the spherical model with some minor corrections to notation and using a slightly different line of reasoning.
	Again we take a lattice $\cL$ of $N$ sites, usually cubic.
	Now our spins at each site are $x_i \in \Omega_0 := \R$ subject to periodic boundary conditions and the \emph{mean spherical constraint} $\sum_{i} x_i^2 = N$. 
	(This constraint is the origin of the name: in the mean, the spins have norm 1 i.e. lie on a unit circle).
	Notice that our spin space is now both infinite, as in the previous section, and not compact.
	We can work with a \emph{grand partition function} given by 
		\[ Q_N(S) = \int_{\R^N} dx_1\ldots d_{x_N} e^{-S\sum\limits_{i} x_i^2+\beta J \sum\limits_{i,j}D_{i,j} x_i x_j+\beta h \sum\limits_i x_i}, \]
	where $\beta$ is the inverse temperature, $J$ is the exchange energy between nearest neighbor spins, $D_{i,j}$ are relative strength of coupling constants, and $h$ is the strength of an external magnetic field, and $S$ implements the mean spherical constraint.
	
	This grand partition function is related to the usual partition function and action by 
		\[ Z = \frac{1}{2\pi i} \int_{\alpha_0-i\infty}^{\alpha_0+i\infty} e^{NS}Q_N(S)\, dS, \]
	where $\alpha_0$ is chosen so that any singularities of the integrand occur to the left of the line $s=\alpha_0$.
	The mean spherical constraint is equivalent to requiring 
		\[N=-\frac{\del}{\del S} \log Q_N(S). \]
	
	Let $n$ be the number of sites in the temporal direction and $\widetilde{N}=N/n$ be the number of sites in any hyperplane slice of $\cL$ with constant temporal coordinate. Now we can take the hyperplane slices of the lattice across one dimension (the temporal dimension) and consider the integral operator $\hat{T}$ defined by 
		\[ \hat{T}f(\vec{x})=\int_{\R^{\widetilde{N}}} L(\vec{x},\vec{y})f(\vec{y}) \, d\vec{y},  \]
	where $\vec{x},\vec{y} \in \R^{\widetilde{N}}$ and $L(\vec{x},\vec{y})$ describes the interactions between hyperplanes.
	We define 
		\begin{equation}
			 L(\vec{x},\vec{y})=e^{-\frac{1}{2}S(\vec{x}^2+\vec{y}^2)+\frac{1}{4}K(\vec{x}^TM\vec{x}+\vec{y}^TM\vec{y})+K_\tau\vec{x}\vec{y}+\frac{1}{2}K'h \sum_{i=1}^{\widetilde{N}} (x_i + y_i)},
		\end{equation}
	where $M$ is a symmetric matrix describing the spin-spin interactions and $K,K_\tau,K'$ are anisotropic coupling constants that may depend on the lattice spacing $\tau$.
	This lets us write $Q_N(S)=\Tr T^n$. As in our earlier examples, we now want to extract a quantum Hamiltonian from this formulation by making appropriate choices of couplings and letting the temporal lattice spacing $\tau$ tend to 0 without solving the problem exactly.
	
	We will need to make use of the operator identity
		\begin{equation}\label{eq:qtfm_20.14_no_limit}
			(e^{-t\hbar\Delta/(2m)}\psi)(\vec{x}_0)=\left( \dfrac{m}{2\pi t \hbar} \right)^{s/2} \int_{\R^s} \exp \left\{-\dfrac{m}{2t\hbar}|\vec{x}_1-\vec{x}_0|^2 \right\}\psi(\vec{x}_1) d\vec{x}_1
		\end{equation}
	which can be found, for example, in \cite[Section 20.3]{hall_qtfm_2013}.
	
	Let $\varphi,\psi \in L^2(\R^{\widetilde{N}})$ and define
	 \begin{align}
		 \widetilde{\varphi}(\vec{y})&=\varphi(\vec{y})e^{-\frac{1}{2}(S(\tau)-K_\tau(\tau))\vec{y}^2+\frac{1}{4}K(\tau)\vec{y}^TM\vec{y}+\frac{1}{2}K'(\tau)h\sum_{i} y_i} \\
		 \widetilde{\psi}(\vec{x})&=\psi(\vec{x})e^{-\frac{1}{2}(S(\tau)-K_\tau(\tau))\vec{x}^2+\frac{1}{4}K(\tau)\vec{x}^TM\vec{x}+\frac{1}{2}K'(\tau)h\sum_{i} x_i}.
	 \end{align}
	 
	We can then write
	\begin{equation}
		\bra{\varphi} \hat{T}_\tau \ket{\psi} = \int d^{\widetilde{N}}\vec{x}\, d^{\widetilde{N}}\vec{y\,} \widetilde{\varphi}(\vec{y}) e^{-\frac{1}{2}K_\tau(\tau)(\vec{x}-\vec{y})^2}\widetilde{\psi}(\vec{x}).
	\end{equation}
	
	Applying \cref{eq:qtfm_20.14_no_limit} to the $\vec{y}$ integral we have
	\begin{align}
		\bra{\varphi} \hat{T}_\tau \ket{\psi} &= \left(\frac{\pi}{2K_\tau} \right)^{\widetilde{N}/2} \int d^{\widetilde{N}}\vec{x}\, \widetilde{\varphi}(\vec{x}) e^{\frac{1}{4K_\tau(\tau)}\Delta}(\vec{x})\widetilde{\psi}(\vec{x}) \\
		&= \left(\frac{\pi}{2K_\tau} \right)^{\widetilde{N}/2} \int d^{\widetilde{N}}\vec{x}\, \widetilde{\varphi}(\vec{x}) ( \mathds{1} + \frac{1}{4K_\tau(\tau)}\Delta + \ldots )\widetilde{\psi}(\vec{x})
	\end{align}
	
	But this implies that $\hat{T}_\tau$ is actually multiplication by 
	\begin{equation}
		\hat{T}_\tau(\vec{x}) = \left(\frac{\pi}{2K_\tau} \right)^{\widetilde{N}/2} e^{-\frac{1}{2}(S-K_\tau)\vec{x}^2+\frac{1}{4}K\vec{x}^TM\vec{x}+\frac{1}{2}K'h\sum_{i} x_i}e^{\frac{1}{4K_\tau}\Delta}(\vec{x})e^{-\frac{1}{2}(S-K_\tau)\vec{x}^2+\frac{1}{4}K\vec{x}^TM\vec{x}+\frac{1}{2}K'h\sum_{i} x_i}.
	\end{equation}
	
	If we expand the exponentials, writing out explicitly the terms that are the product of only the degree 1 and degree 0 terms, this becomes
	\begin{equation}
		\hat{T}_\tau(\vec{x}) = \left(\frac{\pi}{2K_\tau} \right)^{\widetilde{N}/2} \left( 1 -(S(\tau)-K_\tau(\tau))\vec{x}^2+\frac{1}{2}K(\tau)\vec{x}^T M \vec{x} + \frac{1}{2} K'(\tau)h\sum_{i} x_i + \frac{1}{4K_\tau(\tau)}\Delta + \ldots \right).
	\end{equation}
	
	Each of the constants $K, K\tau$, and $K'$ depend implicitly on $\beta$. 
	Separating this out, we can now make a choice of the explicit dependence on $\tau$ so that the terms above are the only terms of the transfer operator involving terms with a power of $\tau$ at most 1.
	In particular, for some constant $\omega$, set
	\begin{equation}
		S(\tau)-2K_\tau(\tau)=\frac{\omega^2}{8}K_\tau(\tau), \quad K(\tau)=\beta \tau, \quad K_\tau(\tau)=\frac{\beta}{4\tau}, \quad K'(\tau)=\beta\tau.
	\end{equation}
	
	Then in the limit $\tau \to 0$ we have 
	\begin{equation}
		\hat{T}_\tau(\vec{x})=(2\pi \tau \beta^{-1})^{\widetilde{N}/2} \left( 1-\frac{\omega^2\tau}{2\beta}\vec{x}^2+\frac{1}{2}\tau\beta \vec{x}^T M \vec{x} +\frac{1}{2}\beta\tau h \sum_i x_i +\frac{\tau}{\beta} \Delta + O(\tau^2) \right).
	\end{equation}
	
	After pulling out constants and approximating a Hamiltonian with $\hat{T}=\mathds{1}-\tau\hat{H}$, we see
	\begin{equation}
		\hat{H} = -\Delta+ \frac{1}{2}(\omega^2\vec{x}^2-\beta^2 \vec{x}^TM\vec{x}-\beta^2 h \sum_i x_i).
	\end{equation}
	%%%%%%%%%%%%%%%%%%%%%%%%%%%%%%%%%%%%%%%%%%%%%%%%%%%%%%%%%%%%%%%
	\section{Lattice Gauge Theories}
	%%%%%%%%%%%%%%%%%%%%%%%%%%%%%%%%%%%%%%%%%%%%%%%%%%%%%%%%%%%%%%%
	
	A \emph{gauge theory} is a physical model which is left invariant under the action of some set of \emph{gauge transformations} at points of the model.
	For example, the electric field is the gradient of electric potential and so is left invariant under translation of the potential by any fixed constant.
	Mathematically, a gauge is a choice of a section of some bundle and a gauge transformation is a map between two such sections.
	In some cases, it can be useful to fix a particular gauge condition to simplify the description of a problem.	
			
	%%%%%%%%%%%%%%%%%%%%%%%%%%%%%%%%%%%%%%%%%%%%%%%%%%%%%%%%%%%%%%%
	\subsection{Ising Lattice Gauge Theory}
	
	Outline/add details to some material from \cite{KogutGaugeSummary}, maybe discuss \cite{MuellerJohnstonJanke17} plaquette solution technique?
	
	The traditional Ising model possesses a global symmetry - if every spin in the lattice is flipped at once, then the value of the action $S$ is preserved.
	On the other hand, if we flip only the spin associated to a single site in the lattice, the terms involving that spin all flip and so the value of $S$ may not be preserved.
	However, we can define a very similar model that is invariant under local transformations as well as global ones.
	
	Let $\cL$ be a $d$-dimensional cubic lattice and label the links of the lattice by an adjacent site $\ell$ and a unit vector in the lattice $\mu$. 
	On each link we place an spin $\sigma_3(\ell,\mu)\in \{\pm 1\}$ and define a \emph{local gauge transformation} $G(\ell)$ to be the operation of flipping the $2d$ spins connected to the site $\ell$.
	Any action which is invariant under such transformations must be constructed from products of $\sigma_3$ on closed paths in the lattice so that any gauge transformation changes an even number of signs in each term of the action.
	For example, the most basic such action is 
		\[S=-J\sum_{\ell,\mu,\nu}\sigma_3(\ell,\mu)\sigma_3(\ell+\mu,\nu)\sigma_3(\ell+\mu+\nu,-\mu)\sigma_3(\ell+\nu,-\nu), \]
	where $\ell$ ranges over all sites, and $\mu,\nu$ over all unordered pairs of distinct unit vectors in the lattice.
	These fundamental squares are called \emph{plaquettes}.
	Sometimes we may omit the indexing for brevity and write
		\[S=-J\sum_{\Box} \sigma_3\sigma_3\sigma_3\sigma_3 \]
	for the sum of the product of spins over all plaquettes in the lattice. 
	Two different configurations $s$ and $\tilde{s}$ of spins are called \emph{gauge equivalent} if there is a sequence $G(\ell_1),G(\ell_2),\ldots$ of gauge transformations so that 
		\[ \tilde{s}=s\cdot G(\ell_1)G(\ell_2)\ldots. \]
	We will allow infinite products of gauge transformations when we work with an infinite lattice.
	
	We may `spend' the gauge symmetry to impose some condition on the configurations we work with and sum only over configurations satisfying this condition.
	As long as we are only interested in observables which are gauge-invariant, this will not affect the resulting value of the observable, since we can write
	\begin{align*}
		\langle \hat{f} \rangle &= \frac{1}{Z}\sum_{\{\sigma\}} \hat{f} (\{\sigma\}) e^{-\beta S(\{\sigma\})} \\
			&= \frac{\sum_{\{\sigma\}} \hat{f} (\{\sigma\}) e^{-\beta S(\{\sigma\})}}{\sum_{\{\sigma\}} e^{-\beta S(\{\sigma\})}}\\
			&= \frac{\sum_{C} \left(\sum_{\{\sigma\}\in C} \hat{f} (\{\sigma\}) e^{-\beta S(\{\sigma\})}\right)}{\sum_{C} \left(\sum_{\{\sigma\}\in C} e^{-\beta S(\{\sigma\})}\right)}\\
			&=\frac{\sum_{C} \left( \hat{f} (\{\sigma_C\}) e^{-\beta S(\{\sigma_C\})}\sum_{\{\sigma\}\in C} 1 \right)}{\sum_{C} \left( e^{-\beta S(\{\sigma_C\})}\sum_{\{\sigma\}\in C} 1\right)} \\
			&=\frac{\sum_{C} \hat{f} (\{\sigma_C\}) e^{-\beta S(\{\sigma_C\})}}{\sum_{C} e^{-\beta S(\{\sigma_C\})}}.
	\end{align*}
	Here $C$ denotes the classes of equivalent configurations subject to the desired constraints.
	Every such class must have the same cardinality, justifying the cancellation in the final line.
	
	If we are working with a lattice which is infinite or has free boundary conditions in the temporal direction, we can apply gauge transformations to pick a representative of any configuration class that satisfies the \emph{temporal gauge}
		\[ \sigma_3(\ell, \tau) =1 \]
	for every link in the temporal direction.	
	Note that this does not yield a unique representative, since any product of gauge transformations that includes all sites with the same spatial coordinates (and varying temporal coordinate) will preserve the temporal gauge on that temporal line.
	In the temporal gauge, we can rewrite the Ising action as
		\[S = -J\sum_{\Box_t} \sigma_3(\ell,\mu)\sigma_3(\ell+\tau,\mu) -J\sum_{\Box_x} \sigma_3\sigma_3\sigma_3\sigma_3,\]
	splitting the sum over the plaquettes involving temporal links and those involving only spatial links.
		
	For example, on a two-dimensional $N_x\times N_t$ lattice with free boundary conditions and unit lattice vectors ($\mu,\tau$), imposing the temporal gauge yields the Ising gauge action
		\[ S=-J\sum_{\ell} \sigma_3(\ell,\mu)\sigma_3(\ell+\tau,\mu). \]
	This is precisely the action of $N_x$ non-interacting copies of a 1-dimensional standard Ising model with free boundary conditions and zero external magnetic field.
	The analysis of this model is very similar to that in Section 2.2 where the model was assumed to have periodic boundary conditions.\\
	
	%TODO - not very happy with this assertion ...
	
	\textbf{Careful:} Under periodic boundary conditions in the temporal direction, gauge transformations can only increase the number of temporal links with $\sigma_3(\ell,\tau)=1$ when applied to sites with $\sigma_3(\ell,\tau)=\sigma_3(\ell,-\tau)=-1$.
	Thus the temporal gauge can be imposed only on the configuration classes consisting of configurations with an \emph{even} number of negative spins on the links of each temporal line with fixed spatial coordinates.
	
	To see this, consider a 1-dimensional chain of $N$ sites with periodic boundary conditions. 
	There are two cases to consider.
	First, any gauge transformation applied to a site where the spins on the temporal links are the same will not change the parity of the number of links with $\sigma_3(\ell,\tau)=1$.
	Second, any gauge transformation applied to a site where the spins on the adjacent temporal links are different will not change the number of links with $\sigma_3(\ell,\tau)=1$.
	From this, it is clear that any configuration with an odd number of links with $\sigma_3(\ell,\tau)=-1$ on the chain cannot be transformed to one with 0 such links.
	
	On the other hand we can show that any configuration with a positive even number of links with $\sigma_3(\ell,\tau)=-1$ is equivalent to one with two fewer such links.
	Suppose that $\sigma_3(\ell,\tau)=-1$ and that the next link with spin $-1$ is $\sigma_3(\ell+k,\tau)$.
	Applying a gauge transformation to the sites $\ell+1,\ell+2,\ldots, \ell+k$ will flip the sign on the links $\sigma_3(\ell,\tau)$ and $\sigma_3(\ell+k,\tau)$ because only one endpoint of the links is acted on.
	For the links between these, both endpoints are acted on, so they are unaffected by the transformation.
	Hence we obtain a configuration with two fewer links with spin $-1$.
	
	With free boundary conditions, there are nodes with only one adjacent temporal link, allowing us to change the number of links with spin $-1$ by 1. \\
	
	TODO - draw diagram to illustrate this
	
	%%%%%%%%%%%%%%%%%%%%%%%%%%%%%%%%%%%%%%%%%%%%%%%%%%%%%%%%%%%%%%%
	\subsection{Lattice $\Z_n$-QED Model in 1-d}
	
	Outline results from \cite{Ercolessi18,Notarnicola15,Wiese13}.
	
	Fix E,V,U convention to be self-consistent/match numerical results in later section of \cite{Ercolessi18}
	
	After more thought, I don't think can have both consistency and agreement with results of \cite{Ercolessi18}.
	
	The standard Lagrangian for QED is 
	\begin{equation}
		\cL = \psi^\dagger\gamma^0[\gamma^\mu(i\del_\mu+gA_\mu)-m]\psi -\frac{1}{4}F_{\mu\nu}F^{\mu\nu}.
	\end{equation}
	
	This can be discretized to get a Hamiltonian on a 1-dimensional spatial lattice \cite{Ercolessi18,Notarnicola15,Wiese13}.
	\begin{equation}\label{eq:1dQEDLatticeHamiltonian}
		H=-\frac{1}{2a} \sum_\ell (\psi_\ell^\dagger U_{\ell,\ell+1}\psi_{\ell+1}+H.c.) + m\sum_\ell (-1)^\ell \psi_\ell^\dagger\psi_\ell+ \frac{g^2 a}{2} \sum_{\ell} E_{\ell,\ell+1}^2,		
	\end{equation}
	where $\ell$ labels the sites of a 1-dimensional lattice with $N$ sites and lattice spacing $a$ and H.c. denotes the Hermitian conjugate of the previous terms. 
	We follow the references in using a staggered mass $(-1)^\ell m$ to avoid the fermion-doubling problem.
	Fermionic matter is represented by the one component spinor creation/annihilation operators $\psi_\ell^\dagger$ and $\psi_\ell$ defined on each site.
	Gauge fields are defined on the links $(\ell,\ell+1)$ of the lattice via electric potential variables $E_{\ell,\ell+1}$ and vector potential variables $A_{\ell,\ell+1}$.
	These give rise to comparators $U_{\ell,\ell+1}(\eta)=e^{-i\eta A_{\ell,\ell+1}}$ and $V_{{\ell,\ell+1}}(\xi)=e^{i\xi E_{\ell,\ell+1}}$ which commute at different links, are parameterized by $\eta,\xi \in \R$, and satisfy the canonical commutation relations
	\begin{equation}\label{eq:1dQEDunitaryCCR}
		V_{\ell,\ell+1}(\xi)U_{\ell,\ell+1}(\eta)=e^{i\eta\xi}U_{\ell,\ell+1}(\eta)V_{\ell,\ell+1}(\xi).
	\end{equation}
	This relation is the exponentiated version of the canonical commutation relation for the $E_{\ell,\ell+1}$ and $A_{\ell,\ell+1}$ variables $[E_{\ell,\ell+1},A_{\ell,\ell+1}]=iI$.
	
	For simulation, we need to have finite quantities everywhere. 
	For this reason, we develop an approximation of the $U(1)$ gauge group by $\Z_n$ via operators that preserve a discrete version of the CCR of \cref{eq:1dQEDunitaryCCR}, following \cite{Ercolessi18}.
	Ercolessi et. al. take a finite-dimensional representation of the two parameter projective unitary Weyl group $\{e^{i(\xi E_{\ell,\ell+1}-\eta A_{\ell,\ell+1})}\}_{\xi,\eta \in \R}$ with the particular choices of parameters
	$(\xi,\eta)=(0,\sqrt{2\pi/n}),(\sqrt{2\pi/n},0)$.
	Making this choice of parameters will give us a discrete version of \cref{eq:1dQEDunitaryCCR} which is periodic in $n^\text{th}$ powers of the operators $U$ and $V$, leading to a representation of $\Z_n$.
	This gives rise to operators
	\begin{equation}\label{eq:1dQED_UVEA_Relations}
		U_{\ell,\ell+1} = e^{-i\sqrt{\frac{2\pi}{n}}A_{\ell,\ell+1}} \qquad V_{\ell,\ell+1} = e^{i\sqrt{\frac{2\pi}{n}}E_{\ell,\ell+1}}
	\end{equation}
	satisfying commutation relations for discrete powers
	\begin{equation}\label{eq:1dQEDDiscreteCCR}
		U_{\ell,\ell+1}^m V_{\ell,\ell+1}^k = e^{i\frac{2\pi}{n} k m} V_{\ell,\ell+1}^k U_{\ell,\ell+1}^m
	\end{equation}
	with $k,m \in \Z_n$.
	
	Alternatively, we can choose parameters $(-2\pi /n,0)$ and $(0,-1)$, giving operators
	\begin{equation}\label{eq:1dQED_better_UVEA_Relations}
		\widetilde{U}_{\ell,\ell+1} = e^{i\widetilde{A}_{\ell,\ell+1}} \qquad \widetilde{V}_{\ell,\ell+1} = e^{-i\frac{2\pi}{n} \widetilde{E}_{\ell,\ell+1}}.
	\end{equation}
	This choice is more convenient later on and the operators still obey the commutation relation of \cref{eq:1dQEDDiscreteCCR}.
	
	We can implement the operators $U$ and $V$ (respectively $\widetilde{U}$ and $\widetilde{V}$) by placing an $n$-dimensional Hilbert space on each link with orthonormal basis $\{\ket{v_k} \}_{0\leq k\leq n-1}$.
	Then $V$ is the diagonal operator given by 
	\begin{equation}
		V\ket{v_k} = e^{-2\pi i k /n} \ket{v_k}.
	\end{equation}
	We define $U$ to be the canonical conjugate operator to $V$, which acts cyclically on the $\ket{v_k}$ basis:
	\begin{equation}
		U\ket{v_k} = \ket{v_{k+1 \Mod{n}}}.
	\end{equation}
	We can check easily that these definitions of $U$ and $V$ do satisfy \cref{eq:1dQEDDiscreteCCR}.
	Further, the homomorphisms from $\Z_n$ taking the generator $1$ to $U$ or $V$ are each faithful representations of $\Z_n$.
	We can also see immediately that $U$ is a unitary operator since $U^\dagger U = U U^\dagger= I$.
	If we express $\mathcal{H}$ in the basis of eigenvectors of $U$, $V$ becomes a shift down operator and so is also unitary.
	Using the convention of \cite{Ercolessi18}, the definition of $V$ together with \cref{eq:1dQED_UVEA_Relations} implies that $E$ must also be diagonal in the $\ket{v_k}$ basis.
	Since the eigenvalues of an exponentiated operator are the exponentials of the original eigenvalues, we have
	\begin{equation}\label{eq:1dQED_EV_eigenvalue_relation}
		e^{-2\pi i k/n} = e^{i\sqrt{2\pi/n} e_k},
	\end{equation}
	where $E\ket{v_k}=e_k \ket{v_k}$ for $k=0,\ldots n-1$.
	Now \cref{eq:1dQED_EV_eigenvalue_relation} implies that 
	\begin{equation}
		e_k=-k'\sqrt{2\pi/n}
	\end{equation}
	for some $k'\equiv k \mod n$.
	
	In \cite{Ercolessi18}, the authors claim that 
		\[e_k=\sqrt{\frac{2\pi}{n}}\left(k-\frac{n-1}{2}+\phi \right)\]
	for some $\phi \in \R$.
	However, this leads to a contradiction unless $n=2$.
	
	Using the second convention and restricting to $n$ odd (for reasons we will see later), we can relabel the basis $\{\ket{v_k}\}_{k=-s}^s$ where $n=2s+1$. 
	Then a possible definition for $\widetilde{E}$ is
	\begin{equation}
		\widetilde{E}\ket{v_k}=k \ket{v_k}.
	\end{equation}
	Notice that this almost implements the relation $[E,U]=U$ satisfied by the original operators in the $U(1)$ formulation:
		\[(\widetilde{E}\widetilde{U}-\widetilde{U}\widetilde{E})\ket{v_k}=\ket{v_{k+1}} \]
	unless $k=s$. 
	In that case $(\widetilde{E}\widetilde{U}-\widetilde{U}\widetilde{E})=-2s\ket{v_{-s}} \neq \ket{v_{-s}} = \widetilde{U}\ket{v_s}$, since our coefficients are not taken mod $n$.
	
	For physical reasons, we need the states of our model to satisfy Gauss's law at each site, which here becomes $G_\ell \ket{\psi}\equiv 0 \Mod{n}$ for each site $\ell$ and
	\begin{equation}\label{eq:1dQED_Gauss_law}
		G_\ell := \psi_\ell^\dagger \psi_\ell + \frac{1}{2}((-1)^\ell -1)-(E_{\ell,\ell+1}-E_{\ell-1,\ell}).
	\end{equation}
	Now this gives us a condition on eigenvalues for physical states. For sites that are occupied by a fermion, $\psi_\ell^\dagger\psi_\ell$ has eigenvalue 1.
	Let $n_\ell=1$ if the site $\ell$ is occupied and $n_\ell=0$ if it is empty.
	Then after a rescaling of the energy to absorb the factor of $\sqrt{2\pi/n}$ into the coupling constants of the Hamiltonian in the Ercolessi setup or directly in our setup, \cref{eq:1dQED_Gauss_law} yields the condition
	\begin{equation}
		n_\ell+\frac{1}{2}((-1)^\ell -1)-(e_R-e_L) \equiv 0 \Mod{n},
	\end{equation}
	where $e_R$ and $e_L$ are the values of $E$ on the links $(\ell,\ell+1)$ and $(\ell-1,\ell)$ respectively.
	Restricting our attention to eigenstates that our physical states are constructed from so that $e_R$ and $e_L$ have the form $k_R$ and $k_L$ for some integers $k_r,k_L \in \Z_n$, we have four cases to consider.
	
	If $\ell$ is an occupied even site, then
		\[ 1+k_L\equiv k_R \Mod{n}. \]
	If $\ell$ is an empty even site, then
		\[k_L \equiv k_R \Mod{n}. \]
	If $\ell$ is an occupied odd site, then
		\[ k_L\equiv k_R \Mod{n}. \]
	If $\ell$ is an empty odd site, then
		\[k_L -1\equiv k_R \Mod{n}. \]
	
	It follows that on a chain of $N$ sites, the value of the field $E$ on the links is completely determined by the value on the first link and whether each site is empty or occupied.
	Thus the physical Hilbert space of the chain is a $(2^N\cdot n)$-dimensional subspace of the full $(2^N\cdot 3^N)$-dimensional (maybe $3^{N+1}$ if open link on both ends) Hilbert space we defined above. 
	
	%%%%%%%%%%%%%%%%%%%%%%%%%%%%%%%%%
	\subsubsection{Parameters of the Hamiltonian}
	
	For a variety of reasons, we may want to rescale parts or all of the terms of our Hamiltonian.
	It may make computations easier or it might make a particular bit of analysis easier.
	To avoid writing out the entire rescaled Hamiltonian each time, we represent these rescalings by a 3-tuple consisting of the coefficients of the hopping term, the mass term, and the electric field term.
	For example, in \cref{eq:1dQEDLatticeHamiltonian} the associated tuple of parameters is $(-1/2a,m,g^2a/2)$.
	
	In particular, it simplifies our analysis if we rescale the Hamiltonian so that the electric field parameter is 1 and the electric field has integer eigenvalues.
	This aligns with the $U(1)$ version of the model studied in \cite{Byrnes02} for example.
	For the Ercolessi setup, this results in the tuple $(\frac{-tn}{g^2a^22\pi},2\frac{nm}{2\pi},1)$ after the authors introduce an extra parameter $t$ affecting only the hopping term.
	For our setup, this results in the tuple $(\frac{-1}{g^2a^2},\frac{2m}{g^2a},1)$.
	To simplify their computational setup, Ercolessi et. al. work with the further rescaled tuple $(\frac{-tn}{2\pi}, 2\tilde{m}\sqrt{n/2\pi}, 1)$.
	They are interested in particular in the value of the critical mass $\tilde{m}$ where the system displays a phase transition and the behavior of this value when $t=1$ and $n$ goes to $\infty$.
	They observe computationally that $\lim_{n\to \infty} \tilde{m}_c \approx 0.33$ for odd $n$.
	
	Now this value is very close to the one obtained in \cite{Byrnes02} for the critical mass for which a phase transition occurs in the continuum limit of the 1-QED model with the full $U(1)$ gauge group.
	But in their analysis, Ercolessi et. al. claim to fix the values $g^2a^2=1, g^2a=2$, which fixes the lattice spacing to be $a=1/2$.
	Thus speaking of a continuum limit does not make sense.
	To reconcile this with the close agreement of the numerical values obtained, we will analyze more closely how the parameters of the two $\Z_n$ setups are related to those of the $U(1)$ model in \cite{Byrnes02}.
	
	The setup of the Hamiltonian in \cite{Byrnes02} mirrors \cref{eq:1dQEDLatticeHamiltonian} exactly except for a factor of $i$ on the hopping term which is absorbed into the exponential in our Hamiltonian.
	In their analysis, they find that the critical value is $m/g=0.3335$.
	If we rewrite our tuple to isolate a factor of $m/g$ we get $(-\gamma,2(m/g)\sqrt{\gamma},1)$ where $\gamma=\frac{1}{g^2a^2}$ correponds to the factor $x$ in \cite{Byrnes02}.
	The claim of \cite{Byrnes02} is then that as we take $\gamma$ to infinity we obtain the continuum limit critical scaled mass value $m/g=0.3335$.
	
	Comparing our rescaled tuple to the computational one used by Ercolessi et. al. we see that in their setup 
		\[\gamma=\frac{n}{2\pi} \qquad  \tilde{m}=(m/g) \qquad 2\sqrt{n/2\pi}=2\sqrt{\gamma}. \]
	
	Substituting the definition of $\gamma$, this shows that the actual lattice spacing must be given by $a=\frac{\sqrt{2\pi}}{\sqrt{n}g}$ and therefore that taking $n$ to infinity does give a continuum limit by forcing $a$ to 0.
	It is less clear why this should work a priori. 
	Why should the lattice spacing need to go to 0 as this particular function of $n$, which is purely a parameter describing the size of the gauge group on the links?
	
	%TODO - consider moving commentary on errors/confusing things in Ercolessi paper to separate section/appendix
	
	%%%%%%%%%%%%%%%%%%%%%%%%%%%%%%%%%%%%%%%%%%%%%%%%%%%%%%%%%%%%%%%
	\subsection{Lattice $\Z_n$-QED Model in 2-d}
	
	Generalize results from \cite{Ercolessi18,Notarnicola15} using general description from \cite{Wiese13}.
	
	Think I have a correct interpretation of Hamiltonian/Gauss Law given in \cite{Wiese13} using constructions from \cite{Ercolessi18,Notarnicola15}. Don't know about reliability given consistency problems of \cite{Ercolessi18}.
	
	In \cite{Wiese13}, Wiese gives a description of the general form of the lattice Hamiltonian for QED on a general $d$-dimensional lattice. Specializing to two dimensions, this becomes
	\begin{align}\label{eq:2dQEDHamiltonian}
		H_{QED}=&-t\sum_{\vec{\ell}} \left( \psi^\dagger_{\vec{\ell}} U_{\vec{\ell},\vec{\ell}+\vec{x}} \psi_{\vec{\ell}+\vec{x}} + (-1)^{\ell_1} \psi^\dagger_{\vec{\ell}}U_{\vec{\ell},\vec{\ell}+\vec{y}} \psi_{\vec{\ell}+\vec{y}} + H.c. \right) + m\sum_{\vec{\ell}} (-1)^{\ell_1+\ell_2} \psi_{\vec{\ell}}^\dagger \psi_{\vec{\ell}} \nonumber \\
			&+ \frac{g^2}{2} \sum_{\vec{\ell}} \left(E^2_{\vec{\ell}+\vec{x}}+E^2_{\vec{\ell}+\vec{y}}\right) - \frac{1}{4g^2} \sum_{\Box} (U_\Box + U_\Box^\dagger).
	\end{align}
	
	Here $\psi$ and $\psi^\dagger$ are anti-commuting annihilation and creation operators on the lattice sites $\vec{\ell}$ describing staggered fermions on the lattice. 
	We say that a site of the 2-dimensional lattice is \emph{odd} (even) if the sum of its coordinates is odd (even).
	The parameters $t,m$, and $g$ describe the hopping action, mass, and energy scaling of the action.
	The operators $E$ and $U$ on the links are related by the commutation relations
	\begin{equation}\label{eq:2dQEDLatticeEUCommutator}
		[E_{e},U_{e'}]=\delta_{e,e'}U_e, \quad [E_{e},U_{e'}^\dagger] = -\delta_{e,e'}U_{e'}^\dagger.
	\end{equation}
	Thus operators on different links commute with each other.
	The plaquette product operator $U_\Box$ is defined by $U_\Box = U_{\vec{\ell},\vec{\ell}+\vec{x}}U_{\vec{\ell}+\vec{x},\vec{\ell}+\vec{x}+\vec{y}}U_{\vec{\ell}+\vec{y},\vec{\ell}+\vec{x}+\vec{y}}^\dagger U_{\vec{\ell},\vec{\ell}+\vec{y}}^\dagger$.
	
	We introduce explicit dependence of the coupling coefficients on the lattice spacing $a$ as in the previous sections in order to obtain the correct continuuum limit.
	The resulting Hamiltonian is
	\begin{align}\label{eq:2dQEDHamiltonianLatticeDepend}
	H_{QED}=&-\frac{1}{2a}\sum_{\vec{\ell}} \left( \psi^\dagger_{\vec{\ell}} U_{\vec{\ell},\vec{\ell}+\vec{x}} \psi_{\vec{\ell}+\vec{x}} + (-1)^{\ell_1} \psi^\dagger_{\vec{\ell}}U_{\vec{\ell},\vec{\ell}+\vec{y}} \psi_{\vec{\ell}+\vec{y}} + H.c. \right) + m\sum_{\vec{\ell}} (-1)^{\ell_1+\ell_2} \psi_{\vec{\ell}}^\dagger \psi_{\vec{\ell}} \nonumber \\
	&+ \frac{g^2a}{2} \sum_{\vec{\ell}} \left(E^2_{\vec{\ell}+\vec{x}}+E^2_{\vec{\ell}+\vec{y}}\right) - \frac{a}{4g^2} \sum_{\Box} (U_\Box + U_\Box^\dagger).
	\end{align}	
	
	As in the one dimensional case, we say that the subspace of eigenstates of $H_{QED}$ obeying Gauss's Law is the \emph{physical Hilbert state}. 
	In this case, the requirement imposed by Gauss's Law is that $G_{\vec{\ell}}\ket{\psi} \equiv 0 \Mod{n}$ for all sites $\ell$, where
	\begin{equation}
		G_{\vec{\ell}}=\psi_{\vec{\ell}}^\dagger\psi_{\vec{\ell}} - ( E_{\vec{\ell},\vec{\ell}+\vec{x}} - E_{\vec{\ell}-\vec{x},\vec{\ell}}) - ( E_{\vec{\ell},\vec{\ell}+\vec{y}} - E_{\vec{\ell}-\vec{y},\vec{\ell}} ).
	\end{equation}
	
	We will actually take a slightly modified version of this so that the Dirac sea, the configuration with 0 electric field, every odd site occupied, and every even site empty, is a physical state:
	\begin{equation}\label{eq:2dQEDGaussLaw}
		G_{\vec{\ell}}=\psi_{\vec{\ell}}^\dagger\psi_{\vec{\ell}} + \frac{1}{2}((-1)^{\ell_1+\ell_2}-1) - ( E_{\vec{\ell},\vec{\ell}+\vec{x}} - E_{\vec{\ell}-\vec{x},\vec{\ell}}) - ( E_{\vec{\ell},\vec{\ell}+\vec{y}} - E_{\vec{\ell}-\vec{y},\vec{\ell}} ).
	\end{equation}
	
	To define a finite approximation to this model, we need to replace the operators $U$ and $E$ with operators in some finite dimensional space. 
	If we replace the gauge group $U(1)$ by $\Z_N$, the commutation relations in \cref{eq:2dQEDLatticeEUCommutator} cannot hold.
	Instead, we preserve an exponentiated version of the canonical commutation relations $[E,A]=iI$ of the $U(1)$ gauge fields.
	Following \cite{Ercolessi18}, \cite{Notarnicola15}, we do this by placing an $n$-dimensional Hilbert space for some odd $n=2s+1\in \N$ on each link with basis $\{\ket{v_k}\}_{k=-s}^{s}$.
	
	We will obtain two different unitary representations of $\Z_n$ on this Hilbert space which implement the commutation relation of \cref{eq:1dQEDDiscreteCCR} as in the one dimensional case.
	Proceeding in the same way, we define operators $U$ and $V$ by
	\begin{equation}
		U\ket{v_k}=\ket{v_{k+1 \Mod{n}}} \qquad V\ket{v_k}= e^{-2\pi i k/n} \ket{v_k}.
	\end{equation}
	Then defining $U=e^{iA}$ and $V=e^{-i2\pi/n E}$ we will let the energy operator $E$ be given by
	\begin{equation}
		E\ket{v_k}=k\ket{v_k}.
	\end{equation}
	
	Since $n$ is odd, there is a non-degenerate unique minimum value of $E^2$ at 0 on each link in this model.
	
	In the 2-dimensional case we have more degrees of freedom in the lattice and the values of the electric field on the links are no longer determined by a single value after applying Gauss's Law.
	Instead, they are determined only partially by the values of the electric field on one link of each vertical and horizontal row of the lattice. 
	Let $e_U,e_D,e_L,e_R$ be the values of the electric field on the four links (up,down,left, right) emanating from a site.
	Then at each site, Gauss's Law in \cref{eq:2dQEDGaussLaw} reduces to the following four equations for eigenstates.
	
	Occupied even site:
		\[ e_R+e_U \equiv 1+e_L+e_D \Mod n. \]
	Empty even site:
		\[ e_R+e_U \equiv e_L+e_D \Mod n. \]
	Occupied odd site:
		\[ e_R+e_U \equiv e_L+e_D \Mod n. \]
	Empty odd site:
		\[ 1+e_R+e_U \equiv e_L+e_D \Mod n. \]
	
	Therefore at any site the values of the electric field on the up and right links satisfy a linear constraint for each possible value of the sum of the values of the down and left links.
	This shows that the values of the electric field on the boundary left and down links of the lattice are free and at each site at the lattice there are a total of $N$ allowed combinations of values of the field on the up/right links.
	Thus the total dimension of the physical Hilbert space on the sites and links of a 2-dimensional $N_X\times N_Y$ lattice in this model is $2^{N_X N_Y} \cdot n^{N_X N_Y + N_X+ N_Y}$.
	For comparison, the dimension of the total Hilbert space is $2^{N_X N_Y} \cdot n^{N_X(N_Y+1)+(N_X+1)N_Y}$.
	So we have the same saving factor of $N^{\text{\# of sites in the lattice}}$ observed in the 1-dimensional case by restricting to physical states.\\
	
	TODO: figure out lattice scaling terms? \\
	\qquad Update 7/9: made a guess that plaquette term coupling should be proportional to $a$ based on analogy to magnetic terms in previous setups.
	Update 7/25: comparing to other literature e.g. \cite{KogutSusskind75} perhaps proportional to $\frac{1}{a}$? \\
	Note plaquette term should have trace in general but this is irrelevant when gauge group is $U(1)$. See \cite{ZoharBurrello15}
	
	
	TODO: draw pictures of lattice gauss law state description?\\
	
	TODO: find references for continuum case/U(1) lattice computations in literature for 2d?
	
	%%%%%%%%%%%%%%%%%%%%%%%%%%%%%%%%%%%%%%%%%%%%%%%%%%%%%%%%%%%%%%%
	\subsection{Formulation of Lattice Gauge Theories}
	
	In \cite{ZoharBurrello15}, Zohar and Burrello describe a general theory for the formulation of Hamiltonians for lattice gauge theories with both matter and gauge fields. 
	Their theory relies heavily on representation theory and so references \cite{hall_lglarep,teleman_rep_theory} may be useful.
	We will summarize their setup and show how their method specializes to the case of 1-d QED seen in the previous sections.
	The theory is valid for any finite group or compact Lie group $G$ used to model the gauge symmetries.
	
	Throughout the section, $g$ denotes elements of $G$, irreducible representations (irreps) of $G$ are indexed by $j$, and the elements of a basis for the associated vector space $V^j$ to the irrep $j$ are $\{\ket{jm} \}_{m=1}^{\dim j}$.
	Without loss of generality, we can restrict to unitary matrix representations of $G$ since it is finite or compact.
	Recall that the \emph{regular representation} of $G$ is the representation given by letting $G$ act on $\C^{G}$ with basis indexed by group elements $g$ such that $g\cdot \ket{h} = \ket{gh}$.
	For finite or compact $G$, the regular representation is completely reducible and decomposes as
	\begin{equation}\label{eq:reg_rep_decomp}
		\C^{G}\cong\bigoplus_j \bigoplus_{i=1}^{\dim (j)} V^j \cong \bigoplus_j V^j \otimes (V^j)^* ,
	\end{equation}
	where the outer sum is over the finite dimensional irreps of $G$ and $(V^j)^*$ is the complex conjugate representation to $V^j$.
	The two bases are related by
	\begin{equation}\label{eq:group_rep_basis_relation}
		\braket{g}{jmn}=\sqrt{\frac{\dim(j)}{|G|}}D^j_{mn}(g),
	\end{equation}
	where $|G|$ is the volume of $G$ with respect to the Haar measure (the counting measure when $G$ is finite).
	We will use the regular representation vector space as the Hilbert space on the links of our model.
	
	The general Hamiltonian proposed by Zohar and Burrello is
	\begin{align}
	H&=\sum_{\ell,\vec{k}} \epsilon_{\ell,\ell+\vec{k}}(\psi_\ell^\dagger U_{\ell,\ell+\vec{k}}^j\psi_{\ell+\vec{k}}+H.c.) + M\sum_\ell (-1)^\ell \psi_\ell^\dagger\psi_\ell \nonumber\\ 
	&- \frac{1}{2t^2} \sum_{\Box} \left(\Tr(U_1^j U_2^j {U_3^j}^\dagger {U_4^j}^\dagger) + H.c.\right) + \frac{t^2}{2} \sum_{(\ell,\ell+1)} \sum_{J} \mathcal{E}(J)\Pi_J.
	\end{align}
	
	We will discuss each term and the operators involved one by one.
	The intial sum is over all lattice sites $\ell$ and all lattice vectors $\vec{k}$.
	First, there is a spinor operator $\psi_\ell$ defined on each site of the lattice.
	We choose a faithful representation $j$ that each spinor and each operator $U^j$ is a part of so that $\psi$ has $\dim(j)$ components $\psi_\ell=(\psi_{\ell,1},\psi_{\ell,2},\ldots,\psi_{\ell,\dim j})$.
	For example, if $G$ is $D_3$, there is only one faithful irreducible representation, so we must use that one (see the next subsection).
	On the other hand, if $G$ is $U(1)$ or $SU(2)$ there are many faithful irreducible representations and we have some freedom to choose which to use.
	The model of Section 5.2 corresponds to the representation $j=1$ of $U(1)$, while the usual model for $SU(2)$ gauge corresponds to the fundamental representation $j=1/2$.
	
	The expression $\psi_\ell^\dagger\psi_m$ is shorthand for the scalar product $\sum_{i=1}^{\dim j} \psi^\dagger_{\ell,i} \psi_{m,i}$.
	To understand how the gauge group acts on $\psi$, let $D^j(g)$ be the matrix representing $g$ in the basis $\{\ket{jm}\}$ of $V^j$.
	Then $g$ acts on $\psi$ and $\psi^\dagger$ as follows:
	\begin{equation}
		\Theta_g^{Q,j} \psi^\dagger \Theta_g^{Q,j \dagger} := g \cdot \psi^\dagger := \psi^\dagger D^j(g) \qquad \Theta_g^{Q,j} \psi \Theta_g^{Q,j \dagger} := g\cdot \psi := D^j(g^{-1}) \psi.
	\end{equation}
	Clearly the term $\psi_\ell^\dagger\psi_\ell$ of the Hamiltonian is left fixed by any such local or global gauge transformation.
	To see that the first term is also invariant under these transformations, we need to describe the $U^j$ operators and their transformation laws.
	(Note that without the $U^j$ operators, the first term is invariant under global gauge transformations but not local ones.)
	
	We can describe the $U^j$ operators in three different ways.
	Examples of each can be found in the following section where we examine the gauge theory for gauge group $D_3$ in detail.
	In each case, the $U^j$ operators act on the space $\C^G \otimes V^j$ formed by taking the tensor product of the defining spaces of the regular representation and the $j$ representation.
	First, we can define the $U^j$ in the basis $\{\ket{g}\otimes\ket{v_i} \}_{g\in G, i=1,\ldots,\dim(j)}$ using the group basis for $\C^G$ and the matrices $D^j(g)$ defining the $j$ representation \cite{TCL14}. 
	\begin{equation}
		U^j=\sum_{g \in G} \left[(\ket{g}\bra{g}) \otimes D^j(g) \right]
	\end{equation}
	
	In this basis, $U^j$ is block diagonal with each block consisting of the matrix $D^j(g)$ if the basis vectors are grouped by the $g$ component. 
	Second, we can follow Zohar and Burrello by grouping the basis vectors by the $v_i$ component and splitting the tensor product into the direct sum $\oplus_{i=1}^{\dim(j)}(\C^G \otimes \langle \ket{v_i}\rangle)$.
	This splits $U^j$ into $\dim(j)^2$ operators $U^j_{mn}: \C^G \otimes \langle \ket{v_n}\rangle \to \C^G \otimes \langle \ket{v_m}\rangle$.
	
	Finally, we can apply a change of basis on each block $\C^G \otimes \langle \ket{v_i}\rangle$ to use the representation basis for $\C^G$.
	To do this, we introduce the Fock space over $\C^G$ with associated raising and lowering ladder operators $a^j_{mn}, a^{j\dagger}_{mn}$, where $j$ indexes an irrep of $G$, $n$ indexes the copy of the irrep appearing in \cref{eq:reg_rep_decomp}, and $m$ indexes the basis vectors of that copy.
	The raising operator takes the local vacuum to the state $\ket{jmn}$: $a^{j\dagger}_{mn} \ket{\Omega} = \ket{jmn}$.
	To avoid introducing extraneous states, we impose the restriction that the number of atoms in the Fock space on any link is always 1: $\sum_{jmn} a^{j\dagger}_{mn} a^j_{mn} =1$.
	This selects a subspace of the Fock space that behaves like a simple Hilbert space.
	On this subspace we can easily describe the action of the ``number-conserving'' operators $a^{J\dagger}_{MN}a^{K}_{M'N'}$.
	\begin{equation}
		a^{J\dagger}_{MN}a^{K}_{M'N'}\ket{jmn} = \delta_{j,K}\delta_{m,M'}\delta_{n,N'}\ket{JMN}.
	\end{equation}
	
	We can now express $U^j_{mn}$ in terms of the ladder operators.
	\begin{proposition}[{\cite[Prop. 3]{ZoharBurrello15}}]\label{prop:Uj_ops}
		The operator elements of $U^j$ are given by
		\[\hat{U}^j_{mm'}= \sum_{J,K} \sqrt{\frac{\dim(J)}{\dim(K)}} \sum_{M,M',N,N'}\braket{J,M,j,m}{K,N} \braket{K, N'}{J,M,j,m'} a^{K\dagger}_{NN'} a^J_{MM'}, \]
		where $J$ and $K$ are irreps of $G$ and $\braket{J,M,j,m}{K,N}, \braket{K, N'}{J,M,j,m'}$ are Clebsch-Gordan coefficients.
	\end{proposition}
	
	Although this sum is infinite for many choice of gauge groups $G$, e.g. $G=U(1)$ or $G=SU(2)$, only finitely many of these terms are nonzero when applied to states $\ket{j'nn'}$ so that there are no convergence issues to worry about.
	We will explicitly compute these operators for the gauge groups $G=D_3$ and $G=SU(2)$ in later sections.
	For now, let's see that this operator agrees with the operator $U$ in the case of $G=U(1)$ seen in the previous sections.
	
	First, recall that the Clebsch-Gordan coefficients describe a change of basis between the tensor product basis of two irreps of $G$ and the decomposition of that tensor product into irreps.
	So the coefficient $\braket{J,M,j,m}{K,N}$ is the coefficient of the basis vector $\ket{JM}\otimes \ket{jm}$ when we write the irrep basis vector $\ket{KN}$ in terms of the tensor product basis.
	Additionally, we are always working with unitary representations and so we can take real Clebsch-Gordan coefficients so that $\braket{J,M,j,m}{K,N}=\braket{K,N}{J,M,j,m}$.
	Knowing how tensoring with the $j$ representation affects other representation will allow us to simplify the expression of \cref{prop:Uj_ops} tremendously.
	For example, tensoring with the trivial representation $j$ gives $\braket{J,M,j,m}{K,N}=\delta_{J,K}\delta_{M,N}$ since this does not change the irrep $J$.
	
	When $G=U(1)$ the irreps are indexed by integers $n\in\Z$ and each is 1-dimensional since $U(1)$ is abelian. 
	The irreps are given by exponentiation to the $n^\text{th}$ power: $\pi_n: U(1) \to GL_1(\C)$ such that $e^{i\theta} \in U(1) \mapsto e^{in\theta} \in GL_1(\C)$.
	We take $j=1$ in the definition of the $U^j$ operator above. 
	Since each irrep is 1-dimensional, we will omit the $m,m',M,M',N,N'$ indices as they are irrelevant. 
	Now $U^j$ will be a $1\times 1$ matrix of operators, i.e. a single operator.
	Tensoring the representations $\pi_n \otimes \pi_1$ gives the the irrep $\pi_{n+1}$. 
	To see this, notice that by definition $\pi_n \otimes \pi_1$ acts on the vector space $\C \otimes_\C \C \cong \C$ by  $(\pi_n\otimes\pi_1)(e^{i\theta})(x\otimes y)=e^{in\theta}x\otimes e^{i\theta}y=e^{i(n+1)\theta}(x\otimes y).$
	It follows that the Clebsch-Gordan coefficients are $\braket{J,1}{K}=\delta_{J+1,K}$.
	Applying this to the proposition we see
	\begin{equation}
		\hat{U^1}=\sum_{J,K} \braket{J,1}{K}\braket{K}{J,1} a^{K\dagger} a^J = \sum_{J} a^{(J+1)\dagger} a^J.
	\end{equation}
	But this operator applied to any basis vector $\ket{J'}$ is simply $\ket{J'+1}$, since $a^J\ket{J'}=\delta_{J,J'}\ket{\Omega}$.
	This is exactly the operator defined in the previous section, so the general theory agrees with what we have seen previously.
	
	These $U^j$ operators transform on the left and right by 
	\begin{equation}
		\Theta_{g}^L U^j \Theta_g^{L\dagger} :=g\cdot U^j := D^j(g^{-1})U^j \qquad \Theta_{g}^R U^j \Theta_g^{R\dagger} := U^j \cdot g := U^j D^j(g).
	\end{equation}
	Applying a gauge transformation to the vertex to the left of an edge acts with the left action and similarly for the vertex on the right of an edge.
	Thus for example the term $\psi_\ell^\dagger U_{\ell,\ell+\vec{k}}^j\psi_{\ell+\vec{k}}$ transforms to $\psi_\ell^\dagger D^j(g) D^j(g^{-1})U_{\ell,\ell+\vec{k}}^j\psi_{\ell+\vec{k}} = \psi_\ell^\dagger U_{\ell,\ell+\vec{k}}^j\psi_{\ell+\vec{k}}$ when the site $\ell$ is acted on by $g$.
	So the hopping term of the Hamiltonian is now locally gauge invariant.
	
	The final term of the proposed Hamiltonian is the ``electric'' term. 
	In it, each $\mathcal{E}(J)$ is an operator that depends only on the representation and $\Pi_J$ is the projection from the link Hilbert space/Fock subspace to the subspace corresponding to only the $J$ representation. 
	In terms of the ladder operators, $\Pi_J=\sum_{mn} a^{J\dagger}_{mn}a^J_{mn}$ since the lowering operators here are 0 on any vector with representation $K\neq J$.
	The coefficient $t$ found on the electric and magnetic (trace of plaquette operator) terms is the gauge field coupling constant. 
	Usually in the literature this is denoted $g$, but we use $t$ to avoid confusion with the use of $g$ as a group element.
	Two examples of the $\mathcal{E}$ operators are the operator $E^2(\ket{j})=j^2$ seen above for QED and the Casimir operator $\mathcal{E}(J)=J(J+1)$ found in the usual treatment of lattice gauge theory for $G=SU(2)$ (see \cite{KogutSusskind75}).
	
	As in Sections 5.2 and 5.3 above, we want to restrict to the subspace of eigenspaces of the Hamiltonian satsifying Gauss's Law.
	In this setting, Gauss's Law says that $\Theta_{g,\ell} \ket{phys} = \ket{phys}$ for any physical state, where
	\begin{equation}
		\Theta_{g,\ell}=\Theta^Q_g\prod_{o} \Theta^L_{g,o} \prod_{i}\Theta^R_{g,i}.
	\end{equation}
	The products are over the links outgoing from and incoming to the site $\ell$ respectively.
	For compact Lie groups, we can rephrase this in terms of infinitesimal generators to arrive at an expression similar to the version of Gauss's Law we saw in earlier sections.
	Define $G_\ell = \sum_{i}R_{g,i}+\sum_{o} L_{g,o} + Q_n$, where
	$\Theta_g^{Q,j}=e^{i\alpha_g \cdot Q^j}, \Theta^L_{g}=e^{i\alpha_g \cdot L}, \Theta^R_g=e^{i\alpha_g \cdot R}$ and $L,R$ obey the commutation relations 
	\begin{align}
		[L_a,L_b]=i\sum_{c}f_{abc}L_c, \quad [R_a,R_b]=i\sum_{c}f_{abc}R_c, \quad [L_a,R_b]=0, \\
		[L_a,U^j_{mn}]=-\sum_{k}(T^j_a)_{mk} U^j_{kn}, \quad [R_a,U^j_{mn}]=\sum_{k}U^j_{mk}(T^j_a)_{kn} . 
	\end{align}
	Here $T^j_a$ is the $j$ representation of the generator $T_a$ of the corresponding Lie algebra.
	Then Gauss's Law is equivalent to requiring that $G_\ell \ket{phys}=0$.
	When $G=U(1)$, we have $L=-R=E$ and $Q=n_\ell - \frac{1}{2}(1-(-1)^n)$ and this is just the Gauss law of Section 5.2.
	
	
	TODO: in what sense is $U^j$ a representation of the group $G$? 
	Gut feeling is that it has to do with tensoring the $j$ representation with the regular representation. 
	This would explain both the dimension of the matrix and why there are Clebsch-Gordan coefficients appearing.
	
	%%%%%%%%%%%%%%%%%%%%%%%%%%%%%%%%%%%%%%%%%%%%%%%%%%%%%%%%%%%%%%%
	\subsection{Lattice $D_3\cong S_3$ Gauge Group Model in 1-d}	
	
	As an example of the general theory of the previous section applied to a finite group, we consider a 1-d lattice with gauge group $G\cong D_3$, extending the example found in \cite[Appendix 1]{ZoharBurrello15}.
	$D_3$ has three irreducible representations.
	Two are one-dimensional: the trivial representation and the sign representation which maps the rotations to $1$ and the reflections to $-1$.
	The last is the defining representation of $D_3$ as $2\times 2$ rotation and reflection matrices:
		\[D_3 \mapsto \left\{ \begin{pmatrix}
		1 & 0\\ 0 & 1
		\end{pmatrix},
		\begin{pmatrix}
		1 & 0\\ 0 & -1
		\end{pmatrix},
		\begin{pmatrix}
		-1/2 & \sqrt{3}/2\\ -\sqrt{3}/2 & -1/2
		\end{pmatrix},
		\begin{pmatrix}
		-1/2 & -\sqrt{3}/2\\ -\sqrt{3}/2 & 1/2
		\end{pmatrix},
		\begin{pmatrix}
		-1/2 & -\sqrt{3}/2\\ \sqrt{3}/2 & -1/2
		\end{pmatrix},
		\begin{pmatrix}
		-1/2 & \sqrt{3}/2\\ \sqrt{3}/2 & 1/2
		\end{pmatrix} \right\} \]
		
	The Hilbert space $\mathcal{H}$ associated to the links is then the 6-dimensional space coming from the regular representation of $D_3$ with basis $\{ \ket{I},\ket{p},\ket{200},\ket{210},\ket{201},\ket{211} \}$.
	Here $\ket{I}$ is the basis vector for the space associated to the trivial irrep, $\ket{p}$ is the basis vector for the parity/sign irrep space, and $\{\ket{200},\ket{210}\}$ and $\{\ket{201},\ket{211}\}$ are bases for the two copies of the defining irrep space.
	
	We start by seeing how this representation basis is related to the group element basis in terms of gives the usual presentation of $D^3$	
		\[\{\ket{e}, \ket{r}, \ket{r^2}, \ket{s}, \ket{rs}, \ket{r^2s}\}\]
 	with $r^3=1, s^2=1$, and $rs=sr^2$.
 	Using \cref{eq:group_rep_basis_relation}, we can write the representation basis vectors as column vectors with respect to the ordered group basis above.
 		\[\ket{I}=\frac{1}{\sqrt{6}}\begin{pmatrix}
 		1 \\ 1 \\ 1 \\ 1 \\ 1 \\ 1
 		\end{pmatrix}\,
 		\ket{p}=\begin{pmatrix}
 		1 \\ 1 \\ 1 \\ -1 \\ -1 \\ -1
 		\end{pmatrix}\,
 		\ket{200}=\frac{1}{\sqrt{3}}\begin{pmatrix}
 		1 \\ -1/2 \\ -1/2 \\ 1 \\ -1/2 \\ -1/2
 		\end{pmatrix}\,
 		\ket{210}=\begin{pmatrix}
 		0 \\ -1/2 \\ 1/2 \\ 0 \\ -1/2 \\ 1/2
 		\end{pmatrix}\,
 		\ket{201}=\begin{pmatrix}
 		0 \\ 1/2 \\ -1/2 \\ 0 \\ -1/2 \\ 1/2
 		\end{pmatrix}\,
 		\ket{211}=\frac{1}{\sqrt{3}}\begin{pmatrix}
 		1 \\ -1/2 \\ -1/2 \\ -1 \\ 1/2 \\ 1/2
 		\end{pmatrix} \]
	As in the general theory, we embed this Hilbert space as a subspace of the Fock space over $\mathcal{H}$ such that for each basis vector there are operators $a,a^\dagger$ so that for example
		\[a^2_{00}\ket{200}=\ket{\Omega} \qquad a^{2\dagger}_{00}\ket{\Omega}=\ket{200}, \]
	where $\ket{\Omega}$ is the vacuum state of the Fock space.
	
	We choose to express the spinor and $U^j$ operators on the links in the two dimensional defining representation.
	Each spinor $\psi_{\ell}$ is thus a vector $(\psi_0,\psi_1)_\ell$.
	
	The Hamiltonian is
	\begin{equation}
		H_{D_3}=h\sum_\ell (\psi_\ell^\dagger U_{\ell,\ell+1}^j\psi_{\ell+1}+H.c.) + M\sum_\ell (-1)^\ell \psi_\ell^\dagger\psi_\ell+ \frac{g^2}{2} \sum_{(\ell,\ell+1)} \sum_{J} \mathcal{E}(J)\Pi_J.
	\end{equation}
	
	The spinors are defined as above. 
	We will express the other operators $U^j, \mathcal{E}(J),\Pi_J$ in terms of the ladder operators $a^J_{mn}$ and ${a^\dagger}^J_{mn}$.
	
	First, we need to understand the Clebsch-Gordan coefficients and thus the tensor products of irreps of $D_3$. 
	There are 6 such tensor products, which decompose into irreps as follows.
	\begin{align}
		V^I \otimes V^I &\cong V^I \qquad V^I \otimes V^p \cong V^p \qquad V^I \otimes V^2 \cong V^2 \nonumber\\
		V^p \otimes V^p &\cong V^I \qquad V^p \otimes V^2 \cong V^2 \qquad V^2 \otimes V^2 \cong V^I \oplus V^p \oplus V^2.
	\end{align}
	The relevant Clebsch-Gordan coefficients are then (taken from \cite{ZoharBurrello15} referencing \cite{vDBC78}):
	\begin{align}
		\braket{I,2m}{2n} &= \delta_{m,n} \qquad \braket{p,2m}{2n} = \epsilon_{m,n} \qquad \braket{2n,2nm}{I}=\delta_{m,n}\frac{1}{\sqrt{2}} \nonumber\\
		\braket{2n,2m}{p} &= \epsilon_{m,n}\frac{1}{\sqrt{2}} \qquad \braket{2n,2m}{2\ell} =\frac{1}{\sqrt{2}}(\delta_{\ell,0}\sigma^z_{nm}-\delta_{\ell,1}\sigma^x_{nm}),
	\end{align}
	where $\sigma^x,\sigma^z$ are the Pauli matrices.
	The other Clebsch-Gordan coefficients appearing in \cref{prop:Uj_ops} are all 0.
	A tedious but straightforward computation shows that the four operators $U^2_{mm'}$ for $m,m' \in \{0,1\}$ are given by the following expressions.
	\begin{align}
		U^2_{00} &= \frac{1}{\sqrt{2}} (a^{2\dagger}_{00}a^I+a^{I\dagger}a^2_{00}+a^{2\dagger}_{11}a^p+a^{p\dagger}a^2_{11} ) +\frac{1}{2} (a^{2\dagger}_{00}a^2_{00}-a^{2\dagger}_{01}a^2_{01}-a^{2\dagger}_{10}a^2_{10}+a^{2\dagger}_{11}a^2_{11}) \nonumber\\
		U^2_{01} &= \frac{1}{\sqrt{2}} (a^{2\dagger}_{01}a^I+a^{I\dagger}a^2_{01}-a^{2\dagger}_{10}a^p-a^{p\dagger}a^2_{10} ) +\frac{1}{2} (-a^{2\dagger}_{01}a^2_{00}-a^{2\dagger}_{00}a^2_{01}+a^{2\dagger}_{11}a^2_{10}+a^{2\dagger}_{10}a^2_{11}) \nonumber\\
		U^2_{10} &= \frac{1}{\sqrt{2}} (a^{2\dagger}_{10}a^I+a^{I\dagger}a^2_{10}-a^{2\dagger}_{01}a^p-a^{p\dagger}a^2_{01} ) +\frac{1}{2} (-a^{2\dagger}_{10}a^2_{00}+a^{2\dagger}_{11}a^2_{01}-a^{2\dagger}_{00}a^2_{10}+a^{2\dagger}_{01}a^2_{11}) \nonumber\\
		U^2_{11} &= \frac{1}{\sqrt{2}} (a^{2\dagger}_{11}a^I+a^{I\dagger}a^2_{11}+a^{2\dagger}_{00}a^p+a^{p\dagger}a^2_{00} ) +\frac{1}{2} (a^{2\dagger}_{11}a^2_{00}+a^{2\dagger}_{10}a^2_{01}+a^{2\dagger}_{01}a^2_{10}+a^{2\dagger}_{00}a^2_{11}) 
	\end{align}
	We can also express these as matrices with respect to the ordered basis $\{ \ket{I},\ket{p},\ket{200},\ket{210},\ket{201},\ket{211} \}$ of the link subspace.
	\begin{align}
		U^2_{00} &= \begin{bmatrix}
		0 & 0 & \frac{1}{\sqrt{2}} & 0 & 0 & 0 \\
		0 & 0 & 0 &0 &0 &\frac{1}{\sqrt{2}}\\
		\frac{1}{\sqrt{2}} & 0 & \frac{1}{2} & 0 & 0 &0 \\
		0 & 0 & 0 & -\frac{1}{2} &0 & 0 \\
		0 &0 &0 &0 &-\frac{1}{2} & 0\\
		0 &\frac{1}{\sqrt{2}} &0 &0 &0 & \frac{1}{2}
		\end{bmatrix}
		\qquad U^2_{01}=\begin{bmatrix}
		0 & 0 & 0 &  0&\frac{1}{\sqrt{2}}  & 0 \\
		0 &0 &0 &-\frac{1}{2}&0  & 0\\
		0 & 0 & 0  &0& -\frac{1}{2} & 0 \\
		0 & -\frac{1}{\sqrt{2}} & 0 & 0 & 0 &\frac{1}{2} \\
		\frac{1}{\sqrt{2}} &0 & -\frac{1}{2}&0 &0 &0 \\
		0 &0 &0 &\frac{1}{2}&0  & 0
		\end{bmatrix}\nonumber\\
		U^2_{10}&= \begin{bmatrix}
		0 &0 &0 &-\frac{1}{\sqrt{2}} &0 & 0\\
		0 & 0 & 0 &  0 &-\frac{1}{\sqrt{2}} & 0 \\
		0 &0 &0 &-\frac{1}{2}&0  & 0\\
		\frac{1}{\sqrt{2}} & 0& -\frac{1}{2} &0 &0 &0 \\
		0 & -\frac{1}{\sqrt{2}} & 0& 0 & 0 &\frac{1}{2} \\
		0 & 0 & 0  &0 & \frac{1}{2}& 0 
		\end{bmatrix}
		\qquad U^2_{11}=\begin{bmatrix}
		0 & 0 & 0 &0 &0 &\frac{1}{\sqrt{2}}\\
		0 & 0 & \frac{1}{\sqrt{2}} & 0 & 0 & 0\\
		0 &\frac{1}{\sqrt{2}} &0 &0 &0 & \frac{1}{2}\\
		0 & 0 & 0 & 0 &\frac{1}{2} & 0\\
		0 & 0 & 0 & \frac{1}{2} &0 & 0 \\
		\frac{1}{\sqrt{2}} & 0 & \frac{1}{2} & 0 & 0 &0 
		\end{bmatrix}
	\end{align}
	
	TODO: What is a ``good'' choice for the electric operator $\mathcal{E}(J)$ for $D_3$ gauge group?
	
	%%%%%%%%%%%%%%%%%%%%%%%%%%%%%%%%%%%%%%%%%%%%%%%%%%%%%%%%%%%%%%%
	\subsection{Lattice $G$-SU(2) Model in 1-d}
	Long term/ambitious: set up lattice Hamiltonian for QCD with SU(2) or SU(3) gauge group and find a sequence of finite groups $G$ to approximate by.
		
	After more research, it may not be possible to approximate $SU(2)$ with a sequence of finite groups to preserve unitarity as in the $\Z_n$ to $U(1)$ case. 
	The following theorem of Toyama says that every compact non-Abelian Lie group cannot be approximated by finite groups in the following sense.
	\begin{theorem}[\cite{toyama1949}]
		Let $G$ be a compact, non-Abelian Lie group. Then there does not exist a sequence of discrete subgroups $(H_n)_{n\geq 0}$ of $G$ such that for every $U \subseteq G$ open there exists $N_U \in \N$ such that for all $n \geq N_U$ we have $U \cap H_n \neq \0$.
	\end{theorem}

	Recall that the irreps of $SU(2)$ are indexed by non-negative half-integers $j=0,1/2,1,3/2,2,\ldots$ and have dimension $2j+1$.In the $G=SU(2)$ case, we will choose the fundamental representation $j=1/2$ for the spinor and $U^j$ operators of Section 5.4.
	For this representation we have $V^J \otimes V^{1/2} \cong V^{J-1/1}\oplus V^{J+1/2}$ for any $J>0$ and $V^0\otimes V^{1/2}\cong V^{1/2}$ since $V^0$ is the trivial representation.
	
	\vspace{0.4cm}

	\textbf{The below material predates section 5.4}. \textit{Leaving it up for now, maybe comment out/delete later.}\\
	
	Don't understand description in \cite{KogutSusskind75,Wiese13} of the formulation of lattice Hamiltonian for SU(2) gauge group.
	In particular, not clear how upper indices of $L,R$ are related to generators of SU(2) or what the defn of L,R ops in \cite{Wiese13} is.
	
	Wiese Hamiltonian for 1-d (no plaquette term) with color indices suppressed:
	
	\begin{equation}
		H_{QCD}=-t\sum_{x} (\psi_x^\dagger U_{x,x+1} \psi_{x+1} + H.c) + m \sum_{x} (-1)^x \psi_x^\dagger\psi_x + \frac{g^2}{2} \sum_{x} (L_{x,x+1}^2+R_{x,x+1}^2),
	\end{equation}
	
	where 
		\[\psi_x^\dagger U_{x,x+1} \psi_{x+1} := \psi_x^{i\dagger} U_{x,x+1}^{ij} \psi_{x+1}^j =?= \sum_i \sum_j \psi_x^{i^\dagger} U_{x,x+1}^{ij} \psi_{x+1}^j,\]
	and the ``color electric field'' is described by the flux operators $L,R$ which obey the following commutation relations for operators on the same link.
		\[ [L^a,L^b]=2if_{abc}L^c, \quad [R^a,R^b]=2if_{abc}R^c, \quad [L^a,R^b]=0, \quad [L^a,U]=-\lambda^a U, \quad [R^a,U]=U\lambda^a. \]
	Here the $\lambda^a,\lambda^b,\lambda^c$ are the Hermitian generators of the $\mathfrak{su(n)}$ algebra with structure constants $f_{abc}$.\\
	
	\noindent Q: What are color indices?\\
	Q: Should the $U$ operators also be color-indexed?\\
	Q: Why are the $L,R$ operators indexed by generators of the $\mathfrak{su(n)}$ algebra?\\
	Q: Are $L,R$ conjugate operators to $U$ like position/momentum?\\
	Q: I think the only difference between $L,R$ is which end of the link they act on/from?\\
	Q: Should other terms also be understood to be sums over color indices? 

	
	%%%%%%%%%%%%%%%%%%%%%%%%%%%%%%%%%%%%%%%%%%%%%%%%%%%%%%%%%%%%%%%
	%%                       References                          %%
	%%%%%%%%%%%%%%%%%%%%%%%%%%%%%%%%%%%%%%%%%%%%%%%%%%%%%%%%%%%%%%%
	\bibliographystyle{abbrv}
	\bibliography{biblio}
	\nocite{Hamer82}
	\nocite{AKHvD11}
\end{document}
