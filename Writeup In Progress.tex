\documentclass[11pt,reqno]{amsart}
\usepackage{amsfonts}
\usepackage{amscd}
\usepackage{amsmath}
\usepackage{amssymb}
\usepackage{amsthm}
\usepackage{enumitem}
\usepackage{pgf,tikz}
\usepackage{mathrsfs}
\usepackage{multicol}
\usepackage[margin=1.25in]{geometry}
\usepackage{etoolbox}
\usepackage{hyperref}
\usepackage{soul}
\usepackage{todonotes}
\usepackage{braket}
\usepackage{physics}
\usepackage[capitalise]{cleveref}
\patchcmd{\section}{\scshape}{\bfseries}{}{}
\makeatletter
\renewcommand{\@secnumfont}{\bfseries}
\makeatother

\newcommand{\Z}{\mathbb{Z}}
\newcommand{\N}{\mathbb{N}}
\newcommand{\ZZ}[1]{\mathbb{Z}/#1\mathbb{Z}}
\newcommand{\R}{\mathbb{R}}
\newcommand{\C}{\mathbb{C}}
\newcommand{\F}{\mathbb{F}}
\newcommand{\Q}{\mathbb{Q}}
\newcommand{\bP}{\mathbb{P}}
\newcommand{\RP}{\mathbb{RP}}
\newcommand{\D}{\mathcal{D}}
\newcommand{\cA}{\mathcal{A}}
\newcommand{\cB}{\mathcal{B}}
\newcommand{\cC}{\mathcal{C}}
\newcommand{\cF}{\mathcal{F}}
\newcommand{\cG}{\mathcal{G}}
\newcommand{\cO}{\mathcal{O}}
\newcommand{\cL}{\mathcal{L}}
\newcommand{\cM}{\mathcal{M}}
\newcommand{\cU}{\mathcal{U}}
\newcommand{\p}{\mathfrak{p}}
\newcommand{\q}{\mathfrak{q}}
\newcommand{\m}{\mathfrak{m}}
\newcommand{\ol}[1]{\overline{#1}}
\newcommand{\Mod}[1]{\ \left(\mathrm{mod}\ #1\right)}
\newcommand{\del}{\partial}
\newcommand{\sm}{\setminus}
\newcommand{\0}{\emptyset}
\newcommand{\cls}[1]{\overline{#1}}
\newcommand{\tsr}[1]{\otimes_{#1}}
\newcommand{\e}{\varepsilon}
\newcommand{\LNorm}[1]{\|#1\|_{L^2[-\pi,\pi]}}
\newcommand{\limup}[1]{\lim_{#1\rightarrow \infty}}
\newcommand{\M}[2]{\mathcal{M}_{#1\times #2}}
\newcommand{\IP}[2]{\langle#1,#2\rangle}
\newcommand*{\fixitem}[1]{\item[]
	\refstepcounter{enumi}\hskip-\labelwidth\hskip-\labelsep
	#1 \labelenumi}
\newcommand{\dbar}{{\mkern3mu\mathchar'26\mkern-12mu d}}

\newtheorem{theorem}{Theorem}
\newtheorem{proposition}[theorem]{Proposition}
\newtheorem{lemma}[theorem]{Lemma}
\newtheorem{conjecture}[theorem]{Conjecture}
\newtheorem{corollary}[theorem]{Corollary}

\theoremstyle{definition}
\newtheorem{example}[theorem]{Example}
\newtheorem{definition}[theorem]{Definition}

\theoremstyle{remark}
\newtheorem*{rem}{Remark}
\newtheorem*{note}{Note}
\newtheorem*{claim}{Claim}

\newcommand{\pder}[2]{\frac{\partial #1}{\partial #2}}
\newcommand{\secpder}[2]{\frac{\partial^2 #1}{\partial #2^2}}
\newsavebox{\smallblockbox}

\newenvironment{smallblockarray}
{\begin{lrbox}{\smallblockbox}
		\scriptsize$\begin{blockarray}}
	{\end{blockarray}$\end{lrbox}%
	\raisebox{-1ex}[\dimexpr\height-2ex][\dimexpr\depth-1ex]{\usebox{\smallblockbox}}}

\newcommand{\module}[2]{%
	\underset{\mathclap{\begin{smallmatrix}\\#2\end{smallmatrix}}}{#1}%
}

\DeclareMathOperator{\Aut}{Aut}
\DeclareMathOperator{\rad}{rad}
\DeclareMathOperator{\Inn}{Inn}
\DeclareMathOperator{\gap}{gap}
\DeclareMathOperator{\lgap}{lgap}
\DeclareMathOperator{\im}{im}
\DeclareMathOperator{\id}{id}
\DeclareMathOperator{\Jac}{Jac}
\DeclareMathOperator{\Int}{Int}
\DeclareMathOperator{\Ann}{Ann}
\DeclareMathOperator{\Ass}{Ass}
\DeclareMathOperator{\ch}{char}
\DeclareMathOperator{\Gal}{Gal}
\DeclareMathOperator{\Tor}{Tor}
\DeclareMathOperator{\Ext}{Ext}
\DeclareMathOperator{\Hom}{Hom}
\DeclareMathOperator{\End}{End}
\DeclareMathOperator{\Spec}{Spec}
\DeclareMathOperator{\ext}{ext}
\DeclareMathOperator{\supp}{supp}
\DeclareMathOperator{\coker}{coker}
\DeclareMathOperator*{\ES}{ES}
\DeclareMathOperator*{\esssup}{ess\,sup}
\DeclareMathOperator{\rk}{rk}
\DeclareMathOperator{\tr}{tr}
\DeclareMathOperator{\eig}{eig}
\DeclareMathOperator{\ev}{ev}
\DeclareMathOperator{\cl}{cl}
\DeclareMathOperator{\Ht}{ht}
\DeclareMathOperator{\inv}{inv}
\DeclareMathOperator{\adj}{adj}
\DeclareMathOperator{\des}{des}
\DeclareMathOperator{\maj}{maj}
\DeclareMathOperator{\Fr}{Fr}
\DeclareMathOperator{\Orb}{Orb}
\DeclareMathOperator{\Stab}{Stab}
\DeclareMathOperator{\Ofr}{OFr}
\DeclareMathOperator{\Map}{Map}
\DeclareMathOperator{\sign}{sign}
\DeclareMathOperator{\lcm}{lcm}
\DeclareMathOperator{\conv}{conv}
\DeclareMathOperator{\Res}{Res}
\DeclareMathOperator{\Log}{Log}
\DeclareMathOperator{\Pic}{Pic}
\DeclareMathOperator{\Div}{Div}
\DeclareMathOperator{\indeg}{indeg}
\DeclareMathOperator{\odeg}{outdeg}
\DeclareMathOperator{\Tr}{Tr}


%\newcommand{\norm}[1]{\left\lVert#1\right\rVert}
\makeatletter
\renewcommand*\env@matrix[1][*\c@MaxMatrixCols c]{%
	\hskip -\arraycolsep
	\let\@ifnextchar\new@ifnextchar
	\array{#1}}
\makeatother
\renewcommand{\Re}{\operatorname{Re}}
\renewcommand{\Im}{\operatorname{Im}}


\endinput


\begin{document}
	
	\title{Notes from NSF-MSGI Internship}
	\author{Hunter Lehmann}
	\maketitle
	
	%%%%%%%%%%%%%%%%%%%%%%%%%%%%%%%%%%%%%%%%%%%%%%%%%%%%%%%%%%%%%%%
	\section{Introduction}
	%%%%%%%%%%%%%%%%%%%%%%%%%%%%%%%%%%%%%%%%%%%%%%%%%%%%%%%%%%%%%%%
	
	We will explore the connection between statistical mechanics models and quantum Hamiltonians, primarily via examples. 
	
	More stuff here later. Basic objects/definitions.
	
	%%%%%%%%%%%%%%%%%%%%%%%%%%%%%%%%%%%%%%%%%%%%%%%%%%%%%%%%%%%%%%%
	\section{Ising Models}
	%%%%%%%%%%%%%%%%%%%%%%%%%%%%%%%%%%%%%%%%%%%%%%%%%%%%%%%%%%%%%%%
	
	The Ising models are a family of statistical mechanics models with nearest-neighbor interaction. 
	Given any lattice $\cL \subset \Z^d$ with $N$ total points, define a \emph{spin} at each site of the lattice via a variable $s \in \Omega_0=\{\pm 1\}$. 
	A \emph{configuration} is a collection $\{s\}=\{s_\ell\}_{\ell\in\cL} \in \Omega_0^N$. 
	For convenience, we will usually work with cubic lattices of the form $\cL=\{-n,-n+1,\ldots, n-1,n\}^d$ or $\cL=\{1\ldots, n\}^d$ and impose periodic boundary conditions. 
	The \emph{action} associated to this model is the function 
	\[\mathcal{S}(\{s\},\tau, h)=-\beta_\tau(\tau)\sum_{i\sim j} s_is_j - \beta(\tau)h\sum_{i}s_i , \] 
	where $\tau$ is the lattice spacing in the ``time" direction, $h$ is the strength of an external magnetic field, and the first sum is over nearest neighbor sites of the lattice. 
	The coefficients $\beta_\tau$ and $\beta$ may depend on the time direction lattice spacing and are otherwise chosen to reflect some physical situation (e.g. they may be taken to be proportional to the ``inverse temperature" $\frac{1}{kT}$ where $k$ is Boltzmann's constant).
	
	We will be interested in the \emph{partition function} $Z$ defined as \[Z(\tau,h)=\sum_{\{s\}\in\Omega_0^N}e^{-\mathcal{S}(\{s\},\tau, h)}. \]
	
	It turns out that we can use statistical mechanics to show that many other interesting physical quantities can be derived from the partition function \cite[Ch. 3]{friedli_velenik_2017}.
	The \emph{pressure in $\cL$} of the model is defined to be 
	\[\psi_{\cL}(\tau,h):=\frac{1}{N} \log Z(\tau,h).  \]
	The \emph{magnetization density in $\cL$} is by definition 
	\[ m_\cL:=\frac{1}{N}\sum_{\ell \in \cL} s_\ell. \]
	The expected magnetization density, $\langle m_\cL \rangle$, is related to the pressure and partition function:
	\[\langle m_\cL \rangle (\tau,h) = \frac{\partial \psi_{\cL}}{\partial h}(\tau, h)=\frac{\partial}{\partial h}\left( \frac{1}{N} \log Z(\tau,h)\right) . \]
	
	
	%%%%%%%%%%%%%%%%%%%%%%%%%%%%%%%%%%%%%%%%%%%%%%%%%%%%%%%%%%%%%%%
	\subsection{Infinite Lattice Ising Models}
	
	The Ising model can be generalized to an \emph{infinite volume model} by letting the lattice $\cL$ grow to $\Z^d$ appropriately. 
	We'll briefly sketch the ideas here - a detailed exposition can be found in \cite{friedli_velenik_2017} in chapters 3 and 6.
	In general, we say a sequence of lattices $\{\cL_n \}$ converges to $\Z^d$, denoted $\cL_n \Uparrow \Z^d$, if 
	\begin{enumerate}
		\item $\cL_n$ is increasing, i.e. $\cL_n \subset \cL_{n+1}$ for all $n$,
		\item $\bigcup_{n} \cL_n = \Z^d$,
		\item $\lim\limits_{n\to \infty} \frac{|\partial \cL_n|}{|\cL_n|}=0$, where $\partial \cL_n$ is the boundary of the lattice, i.e. 
		\[\partial \cL_n := \{\ell \in \cL_n \mid \ell \sim m \, \text{for some $m$}\in \Z^d \sm \cL_n \}. \]
	\end{enumerate}
	It turns out that the pressure $\psi_{\cL}$ defined above is convex as a function of $h$ and that there is a well-defined limit \[\psi(\tau,h):= \lim\limits_{\cL \Uparrow \Z^d} \psi_{\cL}(\tau,h).\]
	Because $\psi$ is a convex function of $h$, the \emph{average magnetization density} given by
	\[m(\tau,h)=\lim_{\cL \Uparrow \Z^d} \langle m_{\cL} \rangle (\tau, h), \]
	exists for all but a countable set of $h \in \R$.
	The points at which the average magnetization density fails to exist (always because the left and right derivatives do not equal each other) are called \emph{first-order phase transitions}.
		
	%%%%%%%%%%%%%%%%%%%%%%%%%%%%%%%%%%%%%%%%%%%%%%%%%%%%%%%%%%%%%%%
	\subsection{1+0 Ising}
	
	In the 1+0 Ising model, we take a 1-dimensional statistical mechanics system and relate it to a 0-dimensional (point) quantum Hamiltonian. See \cite{FradkinSusskind78} and \cite{KogutGaugeSummary}.
	The definitions above simplify to the following. The lattice is now a set of $N$ points $x_0,\ldots,x_{N-1}$ on a circle so that $s_0=s_N$. The action is given by 
	\[\mathcal{S}(\{s\},\tau, h)=-\beta_\tau(\tau)\sum_{i=0}^{N-1} s_is_{i+1} - \beta(\tau)h\sum_{i=0}^{N-1}s_i. \] 
	It will be helpful to rewrite this expression to be symmetric and expressed in terms of sums and differences of the $s_i$. 
	Up to a constant factor of $-\beta_\tau N$, we find
	 \[\mathcal{S}(\{s\},\tau)=\frac{1}{2}\beta_\tau(\tau)\sum_{i=0}^{N-1} (s_i-s_{i+1})^2 - \frac{1}{2}\beta(\tau)h\sum_{i=0}^{N-1}(s_i+s_{i+1}). \]
	Note that we needed periodicity in order to rewrite the second term in this way. Further, we now have $\mathcal{S}=0$ when all of the spins are identical and $h=0$.
	
	Now the partition function becomes 
	\begin{align*}
		Z&=\sum_{\{s\}\in\Omega_0^N} e^{\frac{1}{2}\beta_\tau(\tau)\sum_{i=0}^{N-1} (s_i-s_{i+1})^2 - \frac{1}{2}\beta(\tau)h\sum_{i=0}^{N-1}(s_i+s_{i+1})} \\
		&=\sum_{s_0 \in \{\pm 1\}} \ldots \sum_{s_{N-1}\in \{\pm 1\}} e^{\frac{1}{2}\beta_\tau (s_0-s_1)^2}e^{-\frac{1}{2}\beta h(s_0+s_1)}\cdot\ldots\cdot e^{\frac{1}{2}\beta_\tau (s_{N-1}-s_0)^2}e^{-\frac{1}{2}\beta h(s_{N-1}+s_0)}.
	\end{align*}
	
	However, each factor in the product depends only on what the values of $s_i-s_{i+1}$ and $s_i+s_{i+1}$ are. 
	We can construct the \emph{transfer matrix}, $T$, to record this data, indexed by the possible spins at each site.
	So we have 
	\[T_{-1,-1}=e^{-\beta h}, \quad T_{-1,1}=T_{1,-1}=e^{-2\beta_\tau}, \quad \text{and}\ T_{1,1}=e^{\beta h}. \] 
	Thus \[T=\begin{bmatrix}
	e^{-\beta h} & e^{-2\beta_\tau} \\
	e^{-2\beta_\tau} & e^{\beta h}
	\end{bmatrix}. \]
	
	Now we can write $Z=\Tr T^N$. 
	%TODO: expand on this claim
		
	%TODO: Should I insert more of the action formalism/statisical mechanics stuff here for this model before moving to the Hamiltonian?
	Our goal is to choose functions $\beta_\tau(\tau)$ and $\beta(\tau)$ so that the transfer matrix operator $T$ has the form $T=e^{-\tau H} \approx I-\tau H$ for some quantum Hamiltonian $H$ (independent of $\tau$) acting on a $2$-dimensional vector space. 
	This is the \emph{$\tau$-continuum Hamiltonian} for the model. 
	Here the idea is that statistical mechanics properties of the lattice action system (e.g. magnetization per site, average magnetization, two-point correlations, correlation length) will map to properties of the operator $H$ related to its eigenvalues and eigenvectors.
	
	For example, we could choose $\beta_\tau(\tau)=-\frac{1}{2}\log \tau$ and $\beta(\tau)=\tau$. Then we have \[T=\begin{bmatrix}
	e^{-\tau h} & \tau \\
	\tau & e^{\tau h}
	\end{bmatrix} \approx I_2-\tau\begin{bmatrix}
	h & -1 \\
	-1 & -h
	\end{bmatrix}.\] 
	From this it follows that \[H=\begin{bmatrix}
	h & -1 \\
	-1 & -h
	\end{bmatrix}=-\sigma_1+h\sigma_3, \] where $\sigma_1,\sigma_2$, and $\sigma_3$ are the Pauli matrices. 
	
	More generally, we can ask what constraints $\beta$ and $\beta_\tau$ must satisfy to be able to do this. 
	Analyzing each of the four matrix entries shows that for small $\tau$ we need 
	\[e^{-\beta h} \approx 1-\tau H_{0,0}, \quad e^{-2\beta_\tau} \approx -\tau H_{1,0}=-\tau H_{0,1},\quad \text{and} \ e^{\beta h} \approx 1- \tau H_{1,1}. \]
	
	Since $H_{1,0}$ and $H_{0,1}$ are constant with respect to $\tau$ we need $-2\beta_\tau \approx \log (\lambda\tau)$ for small $\tau$ and some $\lambda\in \R_{>0}$. 
	Similarly, if $\beta$ is small when $\tau$ is small we have $e^{-\beta h} \approx 1-\beta h \approx 1-\tau H_{0,0}$ and $e^{\beta h} \approx 1+\beta h \approx 1-\tau H_{1,1}$. 
	So to first order we have $\beta \approx \tau$ and we see $H_{0,0}=-H_{1,1}=\mu h$ for some $\mu \in \R\sm\{0\}$. 
	
	Putting all of this together, we see that the general $\tau$-continuum Hamiltonian for this model will have the form 
	\[ H=\begin{bmatrix}
	\mu h & -\lambda \\
	-\lambda & -\mu h
	\end{bmatrix}=-\lambda\sigma_1+\mu h\sigma_3.\]
	
	Now we could also attempt to find a second order approximation of $T$ so that $T\approx I - \tau H + \frac{1}{2}\tau^2 H^2$ for an operator $H$ not depending on $\tau$. 
	However, following the approach above will show us that there is no easy solution in this case.  
	To get a higher order approximation (or the actual matrix logarithm solving $T=e^{-\tau H}$ for $H$), we may need to allow $H$ to depend on $\tau$. 
	
	In the second order case, we need 
	\begin{align*}
		e^{-\beta h} &\approx 1- \tau H_{0,0}+\frac{\tau^2}{2}(H_{0,0}^2+H_{0,1}^2),\\
		e^{-2\beta_\tau} &\approx -\tau H_{0,1}+\frac{\tau^2}{2}(H_{0,0}H_{0,1}+H_{0,1}H_{1,1}),\\
		e^{\beta h} &\approx 1 - \tau H_{1,1}+\frac{\tau^2}{2}(H_{0,1}^2+H_{0,0}^2)
	\end{align*}
	for small $\tau$, where we have used the fact that $H$ should be real and symmetric and so $H_{0,1}=H_{1,0}$.
	From the first and third equations, we see that $H_{0,0}$ and $H_{1,1}$ must depend on $h$. 
	But the second equation has no dependence on $h$, so the dependence of $H_{0,1}$ on $h$ must be inversely related to the $H_{0,0}$ and $H_{1,1}$ dependence.
	On the other hand adding the first and third equations yields
	\[\cosh(\beta h) \approx 1-\frac{\tau}{2}( H_{0,0}+ H_{1,1})+\frac{\tau^2}{4}(H_{0,0}^2+2H_{0,1}^2+H_{1,1}^2), \]
	which naturally suggests  taking $\beta=\mu\tau$, $H_{0,0}=-H_{1,1}=\mu h$, and $H_{0,1}=H{1,0}=0$. But this is a contradiction.
	
	%%%%%%%%%%%%%%%%%%%%%%%%%%%%%%%%%%%%%%%%%%%%%%%%%%%%%%%%%%%%%%%
	\subsection{1+1 Ising Model}
	
	We may analyze the $1+1$ Ising model similarly to the $1+0$ model. 
	We are going to relate a statistical mechanics system on a 2-dimensional lattice (with one temporal and one spatial dimension) to a quantum mechanical Hamiltonian on a 1-dimensional system of interacting spins.  
	See \cite{FradkinSusskind78} and \cite{KogutGaugeSummary}.
	
	As in the 1+0 Ising model, we will work with a lattice with periodic boundary conditions. 
	Suppose $\cL$ is a $N_x \times N_\tau$ lattice of points in $\Z^2$ with $\vec{\tau}$ a unit vector in the time direction and $\vec{x}$ a unit vector in the spatial direction. 
	Then the action for the model is 
	\[ \mathcal{S}= -\sum_{\ell} \beta_\tau s_{\ell}s_{\ell+\vec{\tau}}+\beta s_{\ell}s_{\ell+\vec{x}}, \]
	where the sum is over all lattice points $\ell \in \cL$.
	
	We want to write $\mathcal{S}=\sum_{j=1}^{N_\tau} L(j,j+1)$ for some $L(j,j+1)$ that describes the interaction between the spatial rows $j$ and $j+1$.
	To do this, we first rewrite 
	\[\mathcal{S}=\sum_{\ell}\frac{\beta_\tau}{2}(s_{\ell}-s_{\ell+\vec{\tau}})^2 -\frac{1}{2}\beta (s_\ell s_{\ell+\vec{x}}+s_{\ell+\vec{\tau}}s_{\ell+\vec{\tau}+\vec{x}}), \]
	
	using the periodic boundary conditions. 
	Note that the new $\mathcal{S}$ differs from the previous one by a normalization constant so that the first term is 0 when all of the spins are aligned, rather than being $-N\beta_\tau$.
	We can now define 
	\[L(j,j+1):= \sum_{\ell=1}^{N_x} \frac{\beta_\tau}{2} (s_\ell-\tilde{s}_\ell)^2-\frac{\beta}{2}(s_\ell s_{\ell+1}+\tilde{s}_\ell \tilde{s}_{\ell+1}), \]
	where the sum is over the $N_x$ indices in the spatial row, $\{s\}$ is the configuration of row $j$ and $\{\tilde{s}\}$ is the configuration of row $j+1$.
	
	Then the partition function is 
	\[ Z=\sum_{\{s\}} e^{-L(1,2)}e^{-L(2,3)}\ldots e^{-L(N_\tau,1)}. \]
	As in the previous section, we express $Z$ as the trace of the $N_\tau^\text{th}$ power of a transfer matrix $\hat{T}$ describing the transition between rows. 
	Since there are $2^{N_x}$ configurations for each row, $\hat{T}$ will be a $2^{N_x}\times 2^{N_x}$ matrix.
	The elements of $\hat{T}$ can be organized by the number of spin flips between configurations, since these determine the value of the first term of $L(j,j+1)$. We want to find $\beta,\beta_\tau$ so that for $\tau$ near 0, $\hat{T} \approx 1 - \tau \hat{H}$.
	\begin{align*}
		\hat{T}|_{0\,\text{flips}} &= e^{\beta \sum_{\ell=1}^{N_x} s_{\ell} s_{\ell+1}},  \\		
		\hat{T}|_{1\,\text{flip}} &= e^{-2\beta_\tau} e^{\frac{\beta}{2}\sum_{\ell=1}^{N_x}(s_\ell s_{\ell+1}+\tilde{s}_\ell \tilde{s}_{\ell+1})},\\
		\vdots \quad & \qquad\qquad\qquad \vdots \\		
		\hat{T}|_{k\,\text{flips}} &= e^{-2k\beta_\tau} e^{\frac{\beta}{2}\sum_{\ell=1}^{N_x}(s_\ell s_{\ell+1}+\tilde{s}_\ell \tilde{s}_{\ell+1})}		
	\end{align*}
	
	The nicest solution to this is to choose $\beta =\lambda \tau$ and $\tau = e^{-2\beta_\tau}$ for some $\lambda \in \R_{>0}$.
	Then to first order approximation in $\tau$,
	\begin{align*}
		\hat{T}|_{0\,\text{flips}} &\approx 1-\mu\lambda\tau, \\
		\hat{T}|_{1\,\text{flip}} &\approx \tau (1+\kappa \lambda\tau) \approx \tau, \\
		\vdots \quad & \qquad\qquad \vdots \\		
		\hat{T}|_{k\,\text{flips}} &\approx \tau^k (1+\kappa \lambda\tau) \approx 0, 
	\end{align*}
    where $\mu=\sum_{\ell=1}^{N_x}s_{\ell}s_{\ell+1}$ and $\kappa=\frac{1}{2}\sum_{\ell=1}^{N_x} (s_{\ell}s_{\ell+1}+\tilde{s}_{\ell}\tilde{s}_{\ell+1})$ are independent of $\tau$.
    
    We can now express $\hat{T}$ as $I- \tau \hat{H}$ where the Hamiltonian $\hat{H}$ is given in terms of the Pauli operators at each site $\ell$, $\hat{\sigma}_1(\ell)$ and $\hat{\sigma}_3(\ell)$:
    \begin{equation}\label{eq:1+1IsingHamiltonian}
    \hat{H}= -\lambda \sum_{\ell=1}^{N_x} \hat{\sigma}_3(\ell)\hat{\sigma}_3(\ell+1) - \sum_{\ell=1}^{N_x} \hat{\sigma}_1(\ell). 
    \end{equation}
    
    %%%%%%%%%%%%%%%%%%%%%%%%%%%%%%%%%%%%%
    \subsubsection{Jordan-Wigner Transform and Solution in terms of Fermionic operators}
    
    A Hamiltonian of the form of \cref{eq:1+1IsingHamiltonian} can be rewritten in terms of Fermion operators $\{a_j,a^\dagger_j\}_{j=1}^n$ that satisfy the \emph{canonical commutation relations} (CCRs) 
    \[ \{a_j,a_k^\dagger \} = \delta_{k,j}I; \qquad \{a_j,a_k \} =0, \]
    where $\{A,B\} = AB + BA$ is the anticommutator of two operators \cite{KogutGaugeSummary},\cite{nielsen_fermions},\cite{SchultzMattisLieb64}.
    
    We can summarize this transformation as follows:
    \begin{enumerate}
    	\item Use duality to swap the roles of $\hat{\sigma}_1$ and $\hat{\sigma}_3$ in \cref{eq:1+1IsingHamiltonian}.
    	\item Use raising and lowering operators to rewrite $H$ in terms of fermion operators. (Jordan-Wigner transform)
    	\item Convert the resulting operators and quadratic Hamiltonian to momentum space.
    	\item Diagonalize the momentum Fermionic Hamiltonian.
    	\item Determine the eigenvalues of the resulting Hamiltonian.
    \end{enumerate}
	
	%%%%%%%%%%%%%%%%%%%%%%%%%%%%%%%%%%%%%%%%%%%%%%%%%%%%%%%%%%%%%%%
	\section{O(N) Model}
	%%%%%%%%%%%%%%%%%%%%%%%%%%%%%%%%%%%%%%%%%%%%%%%%%%%%%%%%%%%%%%%
	
	These models are extensively treated in \cite{FradkinSusskind78},\cite{HamerKogutSusskind79},\cite{KogutGaugeSummary}. 
	As in the Ising model, we have a lattice $\cL \subset \Z^d$ of $M$ points, usually cubic. 
	The spins at each site $i$ of the lattice are now $\vec{x_i} \in \Omega_0=S^{N-1} = \{\vec{x} \in \R^{N} \mid \|\vec{x}\|=1 \}$. 
	The action is
	\[\mathcal{S}=\sum_{i,j} -J_{i,j}\vec{x_i}\cdot\vec{x_j}, \] where the sum is over nearest-neighbor pairs in $\cL$.
	Frequently we will take the interaction constants $J_{i,j}$ to depend only on which coordinate of the lattice the points differ in.
	As in the earlier examples, it may be helpful to renormalize the action so that when all the $\vec{x_i}$ are equal we get $\mathcal{S}=0$. 
	This results in a renormalized action
	\[\mathcal{S}=\sum_{i,j}\frac{1}{2} J_{i,j}(\vec{x_i}-\vec{x_j})^2. \]
	
	As before, we are interested in thinking of our $d$-dimensional lattice as having $1$ time dimension and $d-1$ spatial dimensions and finding a $\tau$-continuum quantum mechanical Hamiltonian $H$ that corresponds to the action as the lattice spacing $\tau$ goes to $0$.
	Since the spin configuration space is now continuous, the partition function will now be an integral rather than a summation:
	\[ Z=\int_{(S^N)^M} d\vec{x_1}\ldots d\vec{x_M} e^{-\beta\sum_{i,j}\frac{1}{2} J_{i,j}(\vec{x_i}-\vec{x_j})^2}. \]
	
	%%%%%%%%%%%%%%%%%%%%%%%%%%%%%%%%%%%%%%%%%%%%%%%%%%%%%%%%%%%%%%%
	\subsection{O(2) model on a 1-dimensional lattice}
	
	First we consider the $O(2)$ model on a 1-dimensional lattice of $N$ points with periodic boundary conditions and a magnetic field of strength $h$.
	Since our configuration space is $S^1$, we can parameterize the spins $\vec{x_i}=(\cos(\theta_i),\sin(\theta_i))$ for $\theta \in [0,2\pi)$. 
	Under this parameterization, the action becomes
	\begin{align*}
		 \mathcal{S} &= -\beta_\tau \sum_{j=1}^N \cos(\theta_j)\cos(\theta_{j+1})+\sin(\theta_j)\sin(\theta_{j+1}) - \beta h  \sum_{j=1}^N \cos(\theta_j)\\
			&=  - \beta_\tau \sum_{j=1}^N \cos(\theta_j-\theta_{j+1}) - \frac{\beta h}{2}  \sum_{j=1}^N \left(\cos(\theta_j)+\cos(\theta_{j+1})\right). 
	\end{align*}
	
	The corresponding partition function is then
	\[ Z = \int_{[0,2\pi]^N} \left(\prod_{j=1}^N d\theta_j \right) e^{\beta_\tau \sum_{j=1}^N \cos(\theta_j-\theta_{j+1})} e^{\frac{\beta h}{2}  \sum_{j=1}^N \cos(\theta_j)+\cos(\theta_{j+1})}. \]
	We now wish find an analogue of the transfer matrix from the analysis of the Ising model.
	To do this, we need the concept of integral operators.
	Given any $L^2$ function $f(x,y)$ on a domain $E\times E$, we can define the operator $L_f : L^2(E) \to L^2(E)$ by 
	\[(L_f g)(x)=\int_E f(x,y)g(y)\, dy \]
	for any $g \in L^2(E)$.
	
	Define $f(\theta,\phi): [0,2\pi]\times [0,2\pi]$ by 
	\[f(\theta,\phi)=e^{\beta_\tau \cos(\theta -\phi)}e^{\frac{\beta h}{2}(\cos(\theta)+\cos(\phi))}. \]
	Then the integral operator $\hat{T}=L_f$ plays the same role as the transfer matrix above.
	Since $f$ is in $L^2([0,2\pi]^2)$, $\hat{T}$ is trace-class and we have
	\[Z= \int_{[0,2\pi]^N} \left(\prod_{j=1}^N d\theta_j \right) f(\theta_1,\theta_2)\ldots f(\theta_N,\theta_1) = \Tr \hat{T}^N. \]
	
	Again, our new goal is to write $\hat{T}=I-\tau\hat{H}$ when $\tau$ is small for some Hamiltonian on the one site Hilbert space $L^2([0,2\pi])$. 
	Recall that the set $B=\{\psi_m:=\frac{1}{\sqrt{2\pi}} e^{im\theta} \mid m \in \Z \}$ forms an orthonormal Hilbert basis for $L^2([0,2\pi])$.
	We will approximate the action of $\hat{T}$ on elements $\psi_m$ of this basis in order to find our approximate Hamiltonian $\hat{H}$.
	\begin{align}
		(\hat{T}\psi_m)(\theta) &= \frac{1}{\sqrt{2\pi}}\int_{0}^{2\pi} e^{\beta_\tau \cos(\theta -\phi)}e^{\frac{\beta h}{2}(\cos(\theta)+\cos(\phi))} e^{im\phi}\, d\phi \nonumber \\
		&=\frac{1}{\sqrt{2\pi}}e^{\frac{1}{2}\beta h \cos(\theta)} \int_{0}^{2\pi} e^{\beta_\tau \cos(\theta -\phi)}e^{\frac{1}{2}\beta h \cos(\phi)}e^{im\phi}\, d\phi. \label{eq:1+0_O(2)_IntOperator}
	\end{align}
	
	We can use a Fourier transform to rewrite the $\cos(\theta-\phi)$ portion of the exponential in \cref{eq:1+0_O(2)_IntOperator}.
	\begin{equation}\label{eq:FourierTransformExpCos}
		e^{-\beta_\tau+\beta_\tau\cos(\theta-\phi)}=\sum_{\ell\in \Z}e^{i\ell(\theta-\phi)}I_\ell(\beta_\tau),
	\end{equation}
	where $I_\ell(\beta)$ is the Bessel function of imaginary argument.
	In particular, we will let $\beta_\tau=\tau^{-1}$ so that $\beta_\tau$ is large when $\tau$ goes to 0. 
	Then we can approximate $I_\ell(\beta_\tau)$ by the Gaussian $e^{-\ell^2/2\beta_\tau}$ and \cref{eq:FourierTransformExpCos} becomes
	\begin{align}
		e^{\beta_\tau\cos(\theta-\phi)}&\approx e^{\beta_\tau}\sum_{\ell\in \Z}e^{i\ell(\theta-\phi)}e^{-\ell^2/2\beta_\tau}\\
			&=e^{1/\tau}\sum_{\ell\in \Z}e^{i\ell(\theta-\phi)}e^{-\tau\ell^2/2}. \label{eq:FourierTransformExpCosApprox}
	\end{align}
	
	After substituting \cref{eq:FourierTransformExpCosApprox} into \cref{eq:1+0_O(2)_IntOperator}, we can use absolute convergence of the sum to rewrite the expression further.
	\begin{align}
		(\hat{T}\psi_m)(\theta) &\approx \frac{1}{\sqrt{2\pi}}e^{\frac{1}{2}\beta h \cos(\theta)} \int_{0}^{2\pi} e^{1/\tau} \left( \sum_{\ell\in \Z}e^{i\ell(\theta-\phi)}e^{-\tau\ell^2/2} \right) e^{\frac{1}{2}\beta h \cos(\phi)}e^{im\phi}\, d\phi \nonumber \\
			&= \frac{1}{\sqrt{2\pi}}e^{\frac{1}{2}\beta h \cos(\theta)+1/\tau} \sum_{\ell\in \Z} e^{i\ell\theta-\tau\ell^2/2}\left(\int_{0}^{2\pi} e^{\frac{1}{2}\beta h \cos(\phi)}e^{i(m-\ell)\phi}\right) \, d\phi. \label{eq:1+0_O(2)_ApproxIntOperator}
	\end{align}
	
	At this point, we want $\beta\beta_\tau$ to remain finite as $\tau$ goes to 0, so $\beta=\lambda \tau$ for some constant $\lambda$. 
	Thus $\frac{1}{2}\beta h \cos(\phi)$ goes to 0 as $\tau$ goes to 0 and we can expand the corresponding exponential in \cref{eq:1+0_O(2)_ApproxIntOperator}.
	We then use orthogonality to evaluate the resulting integral.
	\begin{align*}
		(\hat{T}\psi_m)(\theta) &\approx \frac{1}{\sqrt{2\pi}}e^{\frac{\lambda\tau h}{2} \cos(\theta)+1/\tau} \sum_{\ell\in \Z} e^{i\ell\theta-\tau\ell^2/2}\left(\int_{0}^{2\pi} (1+\left(\frac{\lambda \tau h}{2}\cos(\phi) \right) e^{i(m-\ell)\phi}\right) \, d\phi,  \\
		&= \frac{1}{\sqrt{2\pi}}e^{\frac{\lambda\tau h}{2} \cos(\theta)+1/\tau} \sum_{\ell\in \Z} e^{i\ell\theta-\tau\ell^2/2}\left( 2\pi \delta_{\ell,m}+ \frac{\lambda \tau h}{2}\int_{0}^{2\pi} \left( \frac{e^{i\theta}+e^{-i\theta}}{2} \right)  e^{i(m-\ell)\phi}\right) \, d\phi,  \\
		&= \frac{1}{\sqrt{2\pi}} e^{\frac{\lambda\tau h}{2} \cos(\theta)+1/\tau} \sum_{\ell\in \Z} 2\pi e^{i\ell\theta-\tau\ell^2/2}\left(\delta_{\ell,m}+ \frac{\lambda \tau h}{4}(\delta_{m+1,\ell}+\delta_{m-1,\ell}) \right),  \\
		&= \sqrt{2\pi} e^{\frac{\lambda\tau h}{2} \cos(\theta)+1/\tau} ( e^{im\theta - \tau m^2/2}+ \frac{\lambda \tau h}{4} (e^{i(m+1)\theta - \tau (m+1)^2/2}+ e^{i(m-1)\theta - \tau (m-1)^2/2} ) ). 
	\end{align*}
	
	At this point, we need to rescale the problem to drop the factor of $e^\frac{1}{\tau}$, as this term goes to infinity as $\tau$ goes to 0.
	After we do this, we will write the action of $\hat{T}$ on $\psi_m$ in terms of the operators $J_z$ and $J_{\pm}$ defined on our basis and extended linearly by 
	\[J_z \psi_m = m \cdot \psi_m, \quad J_{\pm} \psi_m = \psi_{m\pm1}. \]
	To do this we use the fact that $\tau$ is small to expand the remaining exponentials that do not depend on $\theta$ and drop all terms involving a power of $\tau$ greater than one.
	\begin{align}
		(\hat{T}\psi_m)(\theta) &\approx \sqrt{2\pi} \left( 1+\frac{1}{2}\lambda\tau h \cos(\theta)\right) \left( \left( 1- \frac{1}{2}\tau m^2\right) e^{im\theta}+ \frac{\lambda \tau h}{4} \left( e^{i(m+1)\theta}+ e^{i(m-1)\theta} \right) \right) ,\nonumber\\
		&\approx\sqrt{2\pi} \left( \left( 1- \frac{1}{2}\tau m^2+\frac{1}{2}\lambda\tau h \cos(\theta)\right)  e^{im\theta}+ \frac{\lambda \tau h}{4} \left( e^{i(m+1)\theta}+ e^{i(m-1)\theta} \right) \right),\nonumber\\
		&=\sqrt{2\pi} \left( \left( 1- \frac{1}{2}\tau m^2\right)  e^{im\theta}+ \frac{\lambda \tau h}{2} \left( e^{i(m+1)\theta}+ e^{i(m-1)\theta} \right) \right),\nonumber\\
		&=\left( 2\pi \left( I-\tau \left( \frac{J_z^2}{2} + \frac{\lambda h}{2}(J_+ + J_-)\right) \right) \psi_m \right)(\theta).\label{eq:1+0_O(2)_Op_1-tH_Form}
	\end{align}
	
	From \cref{eq:1+0_O(2)_Op_1-tH_Form} we see that the desired Hamiltonian is \[\hat{H}=\pi J_z^2+\pi \lambda h (J_+ + J_-). \]
	
	%%%%%%%%%%%%%%%%%%%%%%%%%%%%%%%%%%%%%%%%%%%%%%%%%%%%%%%%%%%%%%%
	\section{Spherical Model}
	%%%%%%%%%%%%%%%%%%%%%%%%%%%%%%%%%%%%%%%%%%%%%%%%%%%%%%%%%%%%%%%
	
	%%%%%%%%%%%%%%%%%%%%%%%%%%%%%%%%%%%%%%%%%%%%%%%%%%%%%%%%%%%%%%%
	\section{Lattice Gauge Theories}
	%%%%%%%%%%%%%%%%%%%%%%%%%%%%%%%%%%%%%%%%%%%%%%%%%%%%%%%%%%%%%%%
	
	%%%%%%%%%%%%%%%%%%%%%%%%%%%%%%%%%%%%%%%%%%%%%%%%%%%%%%%%%%%%%%%
	%%                       References                          %%
	%%%%%%%%%%%%%%%%%%%%%%%%%%%%%%%%%%%%%%%%%%%%%%%%%%%%%%%%%%%%%%%
	\bibliographystyle{abbrv}
	\bibliography{biblio}
	
	\nocite{hall_qtfm_2013}
	\nocite{Thompson68}
	\nocite{HenkelHoeger84}
	\nocite{nielsen_fermions}
\end{document}