\documentclass[12pt,reqno]{amsart}
\usepackage{amsfonts}
\usepackage{amscd}
\usepackage{amsmath}
\usepackage{amssymb}
\usepackage{amsthm}
\usepackage{enumitem}
\usepackage{pgf,tikz}
\usepackage{mathrsfs}
\usepackage{multicol}
\usepackage[margin=1.25in]{geometry}
\usepackage{etoolbox}
\usepackage{hyperref}
\usepackage{soul}
\usepackage{todonotes}
\usepackage{braket}
\usepackage{physics}
\usepackage[capitalise]{cleveref}
\patchcmd{\section}{\scshape}{\bfseries}{}{}
\makeatletter
\renewcommand{\@secnumfont}{\bfseries}
\makeatother

\newcommand{\Z}{\mathbb{Z}}
\newcommand{\N}{\mathbb{N}}
\newcommand{\ZZ}[1]{\mathbb{Z}/#1\mathbb{Z}}
\newcommand{\R}{\mathbb{R}}
\newcommand{\C}{\mathbb{C}}
\newcommand{\F}{\mathbb{F}}
\newcommand{\Q}{\mathbb{Q}}
\newcommand{\bP}{\mathbb{P}}
\newcommand{\RP}{\mathbb{RP}}
\newcommand{\D}{\mathcal{D}}
\newcommand{\cA}{\mathcal{A}}
\newcommand{\cB}{\mathcal{B}}
\newcommand{\cC}{\mathcal{C}}
\newcommand{\cF}{\mathcal{F}}
\newcommand{\cG}{\mathcal{G}}
\newcommand{\cO}{\mathcal{O}}
\newcommand{\cL}{\mathcal{L}}
\newcommand{\cM}{\mathcal{M}}
\newcommand{\cU}{\mathcal{U}}
\newcommand{\p}{\mathfrak{p}}
\newcommand{\q}{\mathfrak{q}}
\newcommand{\m}{\mathfrak{m}}
\newcommand{\ol}[1]{\overline{#1}}
\newcommand{\Mod}[1]{\ \left(\mathrm{mod}\ #1\right)}
\newcommand{\del}{\partial}
\newcommand{\sm}{\setminus}
\newcommand{\0}{\emptyset}
\newcommand{\cls}[1]{\overline{#1}}
\newcommand{\tsr}[1]{\otimes_{#1}}
\newcommand{\e}{\varepsilon}
\newcommand{\LNorm}[1]{\|#1\|_{L^2[-\pi,\pi]}}
\newcommand{\limup}[1]{\lim_{#1\rightarrow \infty}}
\newcommand{\M}[2]{\mathcal{M}_{#1\times #2}}
\newcommand{\IP}[2]{\langle#1,#2\rangle}
\newcommand*{\fixitem}[1]{\item[]
	\refstepcounter{enumi}\hskip-\labelwidth\hskip-\labelsep
	#1 \labelenumi}
\newcommand{\dbar}{{\mkern3mu\mathchar'26\mkern-12mu d}}

\newtheorem{theorem}{Theorem}
\newtheorem{proposition}[theorem]{Proposition}
\newtheorem{lemma}[theorem]{Lemma}
\newtheorem{conjecture}[theorem]{Conjecture}
\newtheorem{corollary}[theorem]{Corollary}

\theoremstyle{definition}
\newtheorem{example}[theorem]{Example}
\newtheorem{definition}[theorem]{Definition}

\theoremstyle{remark}
\newtheorem*{rem}{Remark}
\newtheorem*{note}{Note}
\newtheorem*{claim}{Claim}

\newcommand{\pder}[2]{\frac{\partial #1}{\partial #2}}
\newcommand{\secpder}[2]{\frac{\partial^2 #1}{\partial #2^2}}
\newsavebox{\smallblockbox}

\newenvironment{smallblockarray}
{\begin{lrbox}{\smallblockbox}
		\scriptsize$\begin{blockarray}}
	{\end{blockarray}$\end{lrbox}%
	\raisebox{-1ex}[\dimexpr\height-2ex][\dimexpr\depth-1ex]{\usebox{\smallblockbox}}}

\newcommand{\module}[2]{%
	\underset{\mathclap{\begin{smallmatrix}\\#2\end{smallmatrix}}}{#1}%
}

\DeclareMathOperator{\Aut}{Aut}
\DeclareMathOperator{\rad}{rad}
\DeclareMathOperator{\Inn}{Inn}
\DeclareMathOperator{\gap}{gap}
\DeclareMathOperator{\lgap}{lgap}
\DeclareMathOperator{\im}{im}
\DeclareMathOperator{\id}{id}
\DeclareMathOperator{\Jac}{Jac}
\DeclareMathOperator{\Int}{Int}
\DeclareMathOperator{\Ann}{Ann}
\DeclareMathOperator{\Ass}{Ass}
\DeclareMathOperator{\ch}{char}
\DeclareMathOperator{\Gal}{Gal}
\DeclareMathOperator{\Tor}{Tor}
\DeclareMathOperator{\Ext}{Ext}
\DeclareMathOperator{\Hom}{Hom}
\DeclareMathOperator{\End}{End}
\DeclareMathOperator{\Spec}{Spec}
\DeclareMathOperator{\ext}{ext}
\DeclareMathOperator{\supp}{supp}
\DeclareMathOperator{\coker}{coker}
\DeclareMathOperator*{\ES}{ES}
\DeclareMathOperator*{\esssup}{ess\,sup}
\DeclareMathOperator{\rk}{rk}
\DeclareMathOperator{\tr}{tr}
\DeclareMathOperator{\eig}{eig}
\DeclareMathOperator{\ev}{ev}
\DeclareMathOperator{\cl}{cl}
\DeclareMathOperator{\Ht}{ht}
\DeclareMathOperator{\inv}{inv}
\DeclareMathOperator{\adj}{adj}
\DeclareMathOperator{\des}{des}
\DeclareMathOperator{\maj}{maj}
\DeclareMathOperator{\Fr}{Fr}
\DeclareMathOperator{\Orb}{Orb}
\DeclareMathOperator{\Stab}{Stab}
\DeclareMathOperator{\Ofr}{OFr}
\DeclareMathOperator{\Map}{Map}
\DeclareMathOperator{\sign}{sign}
\DeclareMathOperator{\lcm}{lcm}
\DeclareMathOperator{\conv}{conv}
\DeclareMathOperator{\Res}{Res}
\DeclareMathOperator{\Log}{Log}
\DeclareMathOperator{\Pic}{Pic}
\DeclareMathOperator{\Div}{Div}
\DeclareMathOperator{\indeg}{indeg}
\DeclareMathOperator{\odeg}{outdeg}
\DeclareMathOperator{\Tr}{Tr}


%\newcommand{\norm}[1]{\left\lVert#1\right\rVert}
\makeatletter
\renewcommand*\env@matrix[1][*\c@MaxMatrixCols c]{%
	\hskip -\arraycolsep
	\let\@ifnextchar\new@ifnextchar
	\array{#1}}
\makeatother
\renewcommand{\Re}{\operatorname{Re}}
\renewcommand{\Im}{\operatorname{Im}}


\endinput


\begin{document}
	
	\title{Notes from NSF-MSGI Internship}
	\author{Hunter Lehmann}
	\maketitle
	
	%%%%%%%%%%%%%%%%%%%%%%%%%%%%%%%%%%%%%%%%%%%%%%%%%%%%%%%%%%%%%%%
	\section{Introduction}
	%%%%%%%%%%%%%%%%%%%%%%%%%%%%%%%%%%%%%%%%%%%%%%%%%%%%%%%%%%%%%%%
	
	We will explore the connection between statistical mechanics models and quantum Hamiltonians, primarily via examples. 
	
	More stuff here later. Basic objects/definitions.
	
	%%%%%%%%%%%%%%%%%%%%%%%%%%%%%%%%%%%%%%%%%%%%%%%%%%%%%%%%%%%%%%%
	\section{Ising Models}
	%%%%%%%%%%%%%%%%%%%%%%%%%%%%%%%%%%%%%%%%%%%%%%%%%%%%%%%%%%%%%%%
	
	The Ising models are a family of statistical mechanics models with nearest-neighbor interaction. 
	Given any lattice $\cL \subset \Z^d$ with $N$ total points, define a \emph{spin} at each site of the lattice via a variable $s \in \Omega_0=\{\pm 1\}$. 
	A \emph{configuration} is a collection $\{s\}=\{s_\ell\}_{\ell\in\cL}$. 
	For convenience, we will usually work with cubic lattices of the form $\cL=\{-n,-n+1,\ldots, n-1,n\}^d$ and impose periodic boundary conditions. 
	The \emph{action} associated to this model is the function 
	\[\mathcal{S}(\{s\},\tau)=-\beta_\tau(\tau)\sum_{i\sim j} s_is_j - \beta(\tau)h\sum_{i}s_i , \] 
	where $\tau$ is the lattice spacing in the ``time" direction, $h$ is the strength of an external magnetic field, and the first sum is over nearest neighbor sites of the lattice. 
	The coefficients $\beta_\tau$ and $\beta$ may depend on the time direction lattice spacing and are otherwise chosen to reflect some physical situation (e.g. they may be taken to be proportional to the ``inverse temperature" $\frac{1}{kT}$ where $k$ is Boltzmann's constant).
	
	We will be interested in the \emph{partition function} $Z$ defined as \[Z=\sum_{\{s\}\in\Omega_0^N}e^{-\mathcal{S}(\{s\},\tau)}. \]
	
	%%%%%%%%%%%%%%%%%%%%%%%%%%%%%%%%%%%%%%%%%%%%%%%%%%%%%%%%%%%%%%%
	\subsection{1+0 Ising}
	
	In the 1+0 Ising model, we take a 1-dimensional statistical mechanics system and relate it to a 0-dimensional (point) quantum Hamiltonian. See \cite{FradkinSusskind78} and \cite{KogutGaugeSummary}.
	The definitions above simplify to the following. The lattice is now a set of $N$ points $x_0,\ldots,x_{N-1}$ on a circle so that $s_0=s_N$. The action is given by 
	\[\mathcal{S}(\{s\},\tau)=-\beta_\tau(\tau)\sum_{i=0}^{N-1} s_is_{i+1} - \beta(\tau)h\sum_{i=0}^{N-1}s_i. \] 
	It will be helpful to rewrite this expression to be symmetric and expressed in terms of sums and differences of the $s_i$. 
	Up to a constant factor of $-\beta_\tau\sum_{i=0}^{N-1}s_i$, we find
	 \[\mathcal{S}(\{s\},\tau)=\frac{1}{2}\beta_\tau(\tau)\sum_{i=0}^{N-1} (s_i-s_{i+1})^2 - \frac{1}{2}\beta(\tau)h\sum_{i=0}^{N-1}(s_i+s_{i+1}). \]
	Note that we needed periodicity in order to rewrite the second term in this way. Further, we now have $\mathcal{S}=0$ when all of the spins are identical and $h=0$.
	
	Now the partition function becomes 
	\begin{align*}
		Z&=\sum_{\{s\}\in\Omega_0^N} e^{\frac{1}{2}\beta_\tau(\tau)\sum_{i=0}^{N-1} (s_i-s_{i+1})^2 - \frac{1}{2}\beta(\tau)h\sum_{i=0}^{N-1}(s_i+s_{i+1})} \\
		&=\sum_{s_0 \in \{\pm 1\}} \ldots \sum_{s_{N-1}\in \{\pm 1\}} e^{\frac{1}{2}\beta_\tau (s_0-s_1)^2}e^{-\frac{1}{2}\beta h(s_0+s_1)}\cdot\ldots\cdot e^{\frac{1}{2}\beta_\tau (s_{N-1}-s_0)^2}e^{-\frac{1}{2}\beta h(s_{N-1}+s_0)}.
	\end{align*}
	
	However, each factor in the product depends only on what the values of $s_i-s_{i+1}$ and $s_i+s_{i+1}$ are. 
	We can construct the \emph{transfer matrix}, $T$, to record this data, indexed by the possible spins at each site.
	So we have 
	\[T_{-1,-1}=e^{-\beta h}, \quad T_{-1,1}=T_{1,-1}=e^{-2\beta_\tau}, \quad \text{and}\ T_{1,1}=e^{\beta h}. \] 
	Thus \[T=\begin{bmatrix}
	e^{-\beta h} & e^{-2\beta_\tau} \\
	e^{-2\beta_\tau} & e^{\beta h}
	\end{bmatrix}. \]
	
	Now we can write $Z=\Tr T^N$. 
	%TODO: expand on this claim
		
	%TODO: Should I insert more of the action formalism/statisical mechanics stuff here for this model before moving to the Hamiltonian?
	Our goal is to choose functions $\beta_\tau(\tau)$ and $\beta(\tau)$ so that the transfer matrix operator $T$ has the form $T=e^{-\tau H} \approx I-\tau H$ for some quantum Hamiltonian $H$ (independent of $\tau$) acting on a $2$-dimensional vector space. 
	This is the \emph{$\tau$-continuum Hamiltonian} for the model. 
	Here the idea is that statistical mechanics properties of the lattice action system (e.g. magnetization per site, average magnetization, two-point correlations, correlation length) will map to properties of the operator $H$ related to its eigenvalues and eigenvectors.
	
	For example, we could choose $\beta_\tau(\tau)=-\frac{1}{2}\log \tau$ and $\beta(\tau)=\tau$. Then we have \[T=\begin{bmatrix}
	e^{-\tau h} & \tau \\
	\tau & e^{\tau h}
	\end{bmatrix} \approx I_2-\tau\begin{bmatrix}
	h & -1 \\
	-1 & -h
	\end{bmatrix}.\] 
	From this it follows that \[H=\begin{bmatrix}
	h & -1 \\
	-1 & -h
	\end{bmatrix}=-\sigma_1+h\sigma_3, \] where $\sigma_1,\sigma_2$, and $\sigma_3$ are the Pauli matrices. 
	
	More generally, we can ask what constraints $\beta$ and $\beta_\tau$ must satisfy to be able to do this. 
	Analyzing each of the four matrix entries shows that for small $\tau$ we need 
	\[e^{-\beta h} \approx 1-\tau H_{0,0}, \quad e^{-2\beta_\tau} \approx -\tau H_{1,0}=-\tau H_{0,1},\quad \text{and} \ e^{\beta h} \approx 1- \tau H_{1,1}. \]
	
	Since $H_{1,0}$ and $H_{0,1}$ are constant with respect to $\tau$ we need $-2\beta_\tau \approx \log (\lambda\tau)$ for small $\tau$ and some $\lambda\in \R_{>0}$. 
	Similarly, if $\beta$ is small when $\tau$ is small we have $e^{-\beta h} \approx 1-\beta h \approx 1-\tau H_{0,0}$ and $e^{\beta h} \approx 1+\beta h \approx 1-\tau H_{1,1}$. 
	So to first order we have $\beta \approx \tau$ and we see $H_{0,0}=-H_{1,1}=\mu h$ for some $\mu \in \R\sm\{0\}$. 
	
	Putting all of this together, we see that the general $\tau$-continuum Hamiltonian for this model will have the form 
	\[ H=\begin{bmatrix}
	\mu h & -\lambda \\
	-\lambda & -\mu h
	\end{bmatrix}=-\lambda\sigma_1+\mu h\sigma_3.\]
	
	Now we could also attempt to find a second order approximation of $T$ so that $T\approx I - \tau H + \frac{1}{2}\tau^2 H^2$ for an operator $H$ not depending on $\tau$. 
	However, following the approach above will show us that there is no easy solution in this case.  
	To get a higher order approximation (or the actual matrix logarithm solving $T=e^{-\tau H}$ for $H$), we may need to allow $H$ to depend on $\tau$. 
	
	In the second order case, we need 
	\begin{align*}
		e^{-\beta h} &\approx 1- \tau H_{0,0}+\frac{\tau^2}{2}(H_{0,0}^2+H_{0,1}^2),\\
		e^{-2\beta_\tau} &\approx -\tau H_{0,1}+\frac{\tau^2}{2}(H_{0,0}H_{0,1}+H_{0,1}H_{1,1}),\\
		e^{\beta h} &\approx 1 - \tau H_{1,1}+\frac{\tau^2}{2}(H_{0,1}^2+H_{0,0}^2)
	\end{align*}
	for small $\tau$, where we have used the fact that $H$ should be Hermitian and so $H_{0,1}=H_{1,0}$.
	From the first and third equations, we see that $H_{0,0}$ and $H_{1,1}$ must depend on $h$. 
	But the second equation has no dependence on $h$, so the dependence of $H_{0,1}$ on $h$ must be inversely related to the $H_{0,0}$ and $H_{1,1}$ dependence.
	On the other hand adding the first and third equations yields
	\[\cosh(\beta h) \approx 1-\frac{\tau}{2}( H_{0,0}+ H_{1,1})+\frac{\tau^2}{4}(H_{0,0}^2+2H_{0,1}^2+H_{1,1}^2), \]
	which naturally suggests  taking $\beta=\mu\tau$, $H_{0,0}=-H_{1,1}=\mu h$, and $H_{0,1}=H{1,0}=0$. But this is a contradiction.
	
	%%%%%%%%%%%%%%%%%%%%%%%%%%%%%%%%%%%%%%%%%%%%%%%%%%%%%%%%%%%%%%%
	\subsection{1+1 Ising Model}
	
	We may analyze the $1+1$ Ising model similarly to the $1+0$ model. 
	We are going to relate a statistical mechanics system on a 2-dimensional lattice (with one temporal and one spatial dimension) to a quantum mechanical Hamiltonian on a 1-dimensional system of interacting spins.
	
	%%%%%%%%%%%%%%%%%%%%%%%%%%%%%%%%%%%%%%%%%%%%%%%%%%%%%%%%%%%%%%%
	\subsection{Infinite Lattice Ising Models}
	
	The Ising model can be generalized to an \emph{infinite volume model} by letting the lattice $\cL$ grow to $\Z^d$ appropriately. 
	We'll briefly sketch the ideas here - a detailed exposition can be found in \cite{friedli_velenik_2017} in chapters 3 and 6. 
	
	%%%%%%%%%%%%%%%%%%%%%%%%%%%%%%%%%%%%%%%%%%%%%%%%%%%%%%%%%%%%%%%
	\section{O(N) Model}
	%%%%%%%%%%%%%%%%%%%%%%%%%%%%%%%%%%%%%%%%%%%%%%%%%%%%%%%%%%%%%%%
	
	These models are extensively treated in \cite{HamerKogutSusskind79} and \cite{FradkinSusskind78}. 
	As in the Ising model, we have a lattice $\cL \subset \Z^d$ of $M$ points, usually cubic. 
	The spins at each site $i$ of the lattice are now $\vec{x_i} \in \Omega_0=S^N = \{\vec{x} \in \R^{N+1} \mid \|\vec{x}\|=1 \}$. 
	The action is
	\[\mathcal{S}=\sum_{i,j} -J_{i,j}\vec{x_i}\cdot\vec{x_j}, \] where the sum is over nearest-neighbor pairs in $\cL$.
	Frequently we will take $J_{i,j}$, the interaction constants, to depend only on which coordinate of the lattice the points differ in.
	As in the earlier examples, it is helpful to renormalize the action so that when all the $\vec{x_i}$ are equal we get $\mathcal{S}=0$. 
	This results in a renormalized action
	\[\mathcal{S}=\sum_{i,j}\frac{1}{2} J_{i,j}(\vec{x_i}-\vec{x_j})^2. \]
	
	As before, we are interested in thinking of our $d$-dimensional lattice as having $1$ time dimension and $d-1$ spatial dimensions and finding a $\tau$-continuum quantum mechanical Hamiltonian $H$ that corresponds to the action as the lattice spacing $\tau$ goes to $0$.
	Since the spin configuration space is now continuous, the partition function will now be an integral rather than a summation:
	\[ Z=\int_{(S^N)^M} d\vec{x_1}\ldots d\vec{x_M} e^{\sum_{i,j}\frac{1}{2} J_{i,j}(\vec{x_i}-\vec{x_j})^2}. \]
	
	%%%%%%%%%%%%%%%%%%%%%%%%%%%%%%%%%%%%%%%%%%%%%%%%%%%%%%%%%%%%%%%
	\section{Spherical Model}
	%%%%%%%%%%%%%%%%%%%%%%%%%%%%%%%%%%%%%%%%%%%%%%%%%%%%%%%%%%%%%%%
	
	%%%%%%%%%%%%%%%%%%%%%%%%%%%%%%%%%%%%%%%%%%%%%%%%%%%%%%%%%%%%%%%
	\section{Lattice Gauge Theories}
	%%%%%%%%%%%%%%%%%%%%%%%%%%%%%%%%%%%%%%%%%%%%%%%%%%%%%%%%%%%%%%%
	
	%%%%%%%%%%%%%%%%%%%%%%%%%%%%%%%%%%%%%%%%%%%%%%%%%%%%%%%%%%%%%%%
	%%                       References                          %%
	%%%%%%%%%%%%%%%%%%%%%%%%%%%%%%%%%%%%%%%%%%%%%%%%%%%%%%%%%%%%%%%
	\bibliographystyle{abbrv}
	\bibliography{biblio}
	
	\nocite{HamerKogutSusskind79}
	\nocite{hall_qtfm_2013}

\end{document}